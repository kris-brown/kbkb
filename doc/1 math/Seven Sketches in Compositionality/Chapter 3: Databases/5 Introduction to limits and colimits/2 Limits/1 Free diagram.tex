% TAG Example
\begin{itemize}
    \item Suppose $\mathcal{J}$ is the free category on the graph $\boxed{1 \rightarrow 2 \leftarrow 3 \rightrightarrows 4 \rightarrow 5}$
    \item We may draw a diagram $\mathcal{J}\xrightarrow{D}\mathcal{C}$ inside $\mathcal{C}$ as below:
    \item \,$\boxed{D_1 \rightarrow D_2 \leftarrow D_3 \rightrightarrows D_4 \rightarrow D_5}$\,
    \item We can represent this diagram as a cone over the diagram by \,picking a $C \in \mathcal{C}$ for which every pair of parallel paths that start from $C$ are the same.
    \item \begin{tikzcd}              & C \arrow[d, "c_3"'] \arrow[ld, "c_1"'] \arrow[ldd, "c_2"']\arrow[rdd, "c_4"] \arrow[rrdd, "c_5"] &               &     \\D_1 \arrow[d] & D_3 \arrow[ld] \arrow[rd] \arrow[rd, bend right]                                                  &               &     \\D_2         &                                                                                                  & D_4 \arrow[r] & D_5\end{tikzcd}\,
  \end{itemize}
