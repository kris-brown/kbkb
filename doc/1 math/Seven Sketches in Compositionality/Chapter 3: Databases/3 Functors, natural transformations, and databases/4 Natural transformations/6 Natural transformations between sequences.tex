% TAG Example
\begin{itemize}
    \item  Any non-decreasing sequence of naturals can be identified with a \href{doc/1 math/Seven Sketches in Compositionality/Chapter 3: Databases/3 Functors, natural transformations, and databases/2 Functors/1 Functor}{functor} $\mathbb{N}\rightarrow \mathbb{N}$, considering the preorder of naturals as a category.
    \item A \href{doc/1 math/Seven Sketches in Compositionality/Chapter 3: Databases/3 Functors, natural transformations, and databases/4 Natural transformations/1 Natural transformation}{natural transformation} between two of these \href{doc/1 math/Seven Sketches in Compositionality/Chapter 3: Databases/3 Functors, natural transformations, and databases/2 Functors/1 Functor}{functors} would require a component $\alpha_n$ for each natural, which means \,a morphism from $F_n \rightarrow G_n$. This exists iff $F(n)\leq G(n)$\,.
    \item Thus we can put a \,preorder\, structure over the \href{doc/1 math/Seven Sketches in Compositionality/Chapter 1: Generative Effects/4 Monotone maps/1 Monotone map}{monotone map} of $\mathbb{N} \rightarrow \mathbb{N}$ (this is a thin \href{doc/1 math/Seven Sketches in Compositionality/Chapter 3: Databases/3 Functors, natural transformations, and databases/4 Natural transformations/3 Functor category}{functor category} $\mathbb{N}^\mathbb{N}$).
  \end{itemize}
