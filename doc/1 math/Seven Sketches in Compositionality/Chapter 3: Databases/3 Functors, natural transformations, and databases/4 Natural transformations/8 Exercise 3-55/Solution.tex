\begin{enumerate}
    \item  \,The individual \href{doc/1 math/Seven Sketches in Compositionality/Chapter 3: Databases/3 Functors, natural transformations, and databases/4 Natural transformations/1 Natural transformation}{natural transformations} satsifying the naturality condition makes the left and right squares commute\,, meaning the whole diagram commutes: \begin{tikzcd}F(c) \arrow[r, "\alpha_c"] \arrow[d, "F(f)"'] & G(c) \arrow[d,"G(f)"] \arrow[r, "\beta_{G(c)}"] & H(c) \arrow[d, "H(G(f))"] \\F(d) \arrow[r, "\alpha_d"']                   & G(d) \arrow[r,"\beta_{G(d)}"]                   & H(d)                     \end{tikzcd}
          \begin{itemize}
            \item Thus the mapping from objects in $F$'s domain to morphisms in $H$'s codomain is given by \,$G;\beta$\,.
          \end{itemize}
    \item Mapping each object to \,its own identity morphism will satisfy the naturality condition (all four edges of the square become identity functions). This will enforce unitality\,.
  \end{enumerate}
