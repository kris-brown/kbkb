% TAG Example
\begin{itemize}
    \item  Pulling back data along a \href{doc/1 math/Seven Sketches in Compositionality/Chapter 3: Databases/3 Functors, natural transformations, and databases/2 Functors/1 Functor}{functor}
    \item We'll migrate data between the graph-indexing schema \textbf{Gr} and the loop schema, called \textbf{DDS} (for discrete-dynamical system).
    \item We'll write down an example instance $\mathbf{DDS}\xrightarrow{I}\mathbf{Set}$
  \end{itemize}

  \begin{minipage}{0.48\textwidth}

    \begin{tabular}{|l|l|}
      \hline
      State & Next \\ \hline
      1     & 4    \\ \hline
      2     & 4    \\ \hline
      3     & 5    \\ \hline
      4     & 5    \\ \hline
      5     & 5    \\ \hline
      6     & 7    \\ \hline
      7     & 6    \\ \hline
    \end{tabular}
  \end{minipage}

  \begin{itemize}
    \item  We want to convert this state information into a graph that will let us visualize our machine.
    \item Use the following \href{doc/1 math/Seven Sketches in Compositionality/Chapter 3: Databases/3 Functors, natural transformations, and databases/2 Functors/1 Functor}{functor} $F$: \begin{tikzcd}\overset{Arr}\bullet \arrow[red,d, "tar", bend left] \arrow[blue,d, "src"', bend right] &\overset{State}\bullet \arrow[red, loop below, "next"] \\\overset{Vert}\bullet&\end{tikzcd}
          \begin{itemize}
            \item \emph{src} is sent to identity
            \item Can now generate a graph using the composite \href{doc/1 math/Seven Sketches in Compositionality/Chapter 3: Databases/3 Functors, natural transformations, and databases/2 Functors/1 Functor}{functor} $\mathbf{Gr}\xrightarrow{F}\mathbf{DDS}\xrightarrow{I}\mathbf{Set}$
          \end{itemize}
  \end{itemize}

  \begin{minipage}{0.48\textwidth}

    \begin{tabular}{|l|l|l|}
      \hline
      Arr & src & tar \\ \hline
      1   & 1   & 4   \\ \hline
      2   & 2   & 4   \\ \hline
      3   & 3   & 5   \\ \hline
      4   & 4   & 5   \\ \hline
      5   & 5   & 5   \\ \hline
      6   & 6   & 7   \\ \hline
      7   & 7   & 6   \\ \hline
    \end{tabular}
  \end{minipage}
  $Vert = \bar{7}$

  \begin{itemize}
    \item  We can now draw the graph: \,\begin{tikzcd}& 1 \arrow[rd] && 2 \arrow[ld]&\\3 \arrow[rd] && 4 \arrow[ld] &&\\& 5&& 6 \arrow[r, bend left] & 7 \arrow[l, bend left]\end{tikzcd}\,
    \item This procecure can be called ``pulling back data along a \href{doc/1 math/Seven Sketches in Compositionality/Chapter 3: Databases/3 Functors, natural transformations, and databases/2 Functors/1 Functor}{functor}". We pulled back data $I$ along functor $F$ (via functor composition).

  \end{itemize}