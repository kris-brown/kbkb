\begin{itemize}
    \item  Central theme of category theory: study of structures and structure-preserving maps.
    \item  Asking which aspects of structure one wants to preserve becomes the question "what category are you working in?".
    \item    Example: there are many functions $\mathbb{R} \xrightarrow{f} \mathbb{R}$, which we can think of observations (rather than view $x$ directly we only view $f(x)$). Only some preserve the order of numbers, only some preserve distances between numbers.
    \item  The less structure that is preserved by our observation of a system, the more 'surprises' when we observe its operations - call these \emph{generative effects}.
    \item  Consider a world of systems which are points which may or may not be connected. There are 5 partitionings or systems of three points.
    \item  Suppose Alice makes observations on systems with a function $\phi$ which returns whether or not points are connected. Alice also has an operation on two systems called \emph{join} which puts two points in the same partition if they are connected in either of the original systems.
    \item  Alice's operation is not preserved by the \emph{join} operation.
    \item  Application: Alice is trying to address a possible contagion and needs to know whether or not it is safe to have each region extract their data and then aggregate vs aggregating data and then extracting from it.
  \end{itemize}