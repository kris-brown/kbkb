\begin{itemize}
    \item Category theory emphasizes that \href{doc/1 math/Seven Sketches in Compositionality/1 Generative Effects/Preorders/Preorder}{preorders} themselves (each a minature world, composed of many relationships) can be related to one another.
    \item \href{doc/1 math/Seven Sketches in Compositionality/1 Generative Effects/4 Monotone maps}{Monotone map} are important because they are the right notion of \emph{structure-preserving map} for preorders.
    \item The map (`cardinality') sending a power-set (with inclusion ordering) to the natural numbers with standard ordering is a \href{doc/1 math/Seven Sketches in Compositionality/1 Generative Effects/4 Monotone maps}{monotone map}.
    \item Given a \href{doc/1 math/Seven Sketches in Compositionality/1 Generative Effects/Preorders/Preorder}{preorder}, the inclusion map of the \href{doc/1 math/Seven Sketches in Compositionality/1 Generative Effects/3 Preorders/Upper set}{upper sets} of $P$ (ordered by inclusion) to the power set of $P$ (ordered by inclusion) is a \href{doc/1 math/Seven Sketches in Compositionality/1 Generative Effects/4 Monotone maps}{monotone map}.

  \end{itemize}