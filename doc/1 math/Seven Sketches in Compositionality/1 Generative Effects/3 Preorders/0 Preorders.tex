\begin{itemize}
    \item Preorders are just equivalence relations without the symmetric condition.
    \item Every set can be considered as a discrete preorder with the \href{doc/1 math/Seven Sketches in Compositionality/1 Generative Effects/2 What is order/Function}{binary relation} of equality. Also the trivial opposite (codiscrete preorder) where all pairs are in the relation.
    \item Every graph yields a preorder on the vertices where $v \leq w$ iff there is a path from $v$ to $w$.
          \begin{itemize}
            \item Reflexive because of length-0 paths, transitive because of path concatenation.
          \end{itemize}
    \item Product of two preorders can be considered as a preorder by only comparing things when both preorders independently agree on the pairs.
  \end{itemize}