\begin{itemize}
    \item Think of a \href{doc/1 math/Seven Sketches in Compositionality/Chapter 2: Resource theories/5 Computing presented V-categories with matrix mult/2 Quantales/1 Quantale}{unital, commutative, quantale} as a kind of navigator.
          \begin{itemize}
            \item A navigator understands 'getting from one place to another'
            \item Different navigators understand different aspects (whether one can get from \emph{A} to \emph{B}, how much time it will take, ...)
            \item What they share in common is that, given routes \emph{A} to \emph{B} and \emph{B} to \emph{C}, they understand how to get a route \emph{A} to \emph{C}.
          \end{itemize}
    \item Because of \href{doc/1 math/Seven Sketches in Compositionality/Chapter 2: Resource theories/5 Computing presented V-categories with matrix mult/2 Quantales/3 All joins implies all meets}{Proposition 2.96}, a quantale has all \href{doc/1 math/Seven Sketches in Compositionality/Chapter 1: Generative Effects/5 Meets and joins/1 Definition and basic examples/Meet and join}{meets and joins} (even though we define it simply as having all joins).
  \end{itemize}
