% TAG Example
\begin{itemize}
    \item What would a \href{doc/1 math/Seven Sketches in Compositionality/Chapter 2: Resource theories/5 Computing presented V-categories with matrix mult/1 Monoidal closed preorders/1 Closed SMP}{monoidal closed structure} mean for the resource theory of chemistry?
    \item For any two material collections, one can form a material collection $c \multimap d$ with the property that, for any \emph{a} one has $a + c \rightarrow d$ iff $a \rightarrow (c \multimap d)$
    \item Concretely, say we have $2 H_2O + 2 Na \rightarrow 2 NaOH + H_2$, we must also have $2H_2O \rightarrow (2Na \multimap (2NaOH+H_2))$
    \item From two molecules of water, you can form a certain substance. This doesn't make sense, so maybe this \href{doc/1 math/Seven Sketches in Compositionality/Chapter 2: Resource theories/2 Symmetric monoidal preorders/1 Definition and first examples/1 Symmetric monoidal structure on a preorder}{symmetric monoidal preorder} is not \href{doc/1 math/Seven Sketches in Compositionality/Chapter 2: Resource theories/5 Computing presented V-categories with matrix mult/1 Monoidal closed preorders/1 Closed SMP}{closed}.
    \item Alternatively, think of the substance $2Na \multimap (2NaOH+H_2)$ as a \emph{potential reaction}, that of converting sodium to sodium-hyroxide+hydrogen. Two molecules of water unlock that potential. \textcolor{white}{NOCARD}
  \end{itemize}