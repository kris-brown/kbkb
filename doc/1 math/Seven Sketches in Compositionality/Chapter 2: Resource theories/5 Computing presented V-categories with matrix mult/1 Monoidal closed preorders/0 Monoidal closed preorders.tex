\begin{itemize}
    \item  The term \emph{closed} in the context of \emph{symmetric monoidal closed} preorders refers to the fact that a hom-element can be constructed from any two elements (the preorder is closed under the operation of "taking homs").
    \item Consider the hom-element $v \multimap w$ as a kind of "single use \emph{v} to \emph{w} converter"
          \begin{itemize}
            \item \emph{a} and \emph{v} are enough to get to \emph{w} iff \emph{a} is enough to get to a single-use \emph{v}-to-\emph{w} converter.
          \end{itemize}
    \item One may read
          \begin{itemize}
            \item \href{doc/1 math/Seven Sketches in Compositionality/Chapter 2: Resource theories/5 Computing presented V-categories with matrix mult/1 Monoidal closed preorders/5 SMP currying}{P2.87c} as saying "if I have a \emph{v} and a single-use \emph{v} to \emph{w} converter, then I have a \emph{w}" and
            \item \href{doc/1 math/Seven Sketches in Compositionality/Chapter 2: Resource theories/5 Computing presented V-categories with matrix mult/1 Monoidal closed preorders/5 SMP currying}{P2.87d} as saying "Having \emph{v} is the same as shaving a single-use nothing-to-\emph{v} converter"
            \item \href{doc/1 math/Seven Sketches in Compositionality/Chapter 2: Resource theories/5 Computing presented V-categories with matrix mult/1 Monoidal closed preorders/5 SMP currying}{P2.87e} as saying "We can compose two single-use converters"
          \end{itemize}
  \end{itemize}
