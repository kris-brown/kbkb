\begin{itemize}
    \item[a] The \href{doc/1 math/Seven Sketches in Compositionality/Chapter 1: Generative Effects/6 Galois connections/1 Definition and examples/Galois connection}{meaning} of $(- \otimes v) \dashv (v \multimap -)$ is \,exactly the condition of $\multimap$ in a \href{doc/1 math/Seven Sketches in Compositionality/Chapter 2: Resource theories/5 Computing presented V-categories with matrix mult/1 Monoidal closed preorders/1 Closed SMP}{closed symmetric monoidal preorder}\,.
    \item[b] This follows from (1), using the fact that \,\href{doc/1 math/Seven Sketches in Compositionality/Chapter 1: Generative Effects/6 Galois connections/3 Basic theory of Galois connections/2 Adjoints preserving meets and joins}{left adjoints preserve joins}\,.
    \item[c] This follows from (1) using \,the \href{doc/1 math/Seven Sketches in Compositionality/Chapter 1: Generative Effects/6 Galois connections/3 Basic theory of Galois connections/1 Galois connection alternate form}{alternative characterization} of \href{doc/1 math/Seven Sketches in Compositionality/Chapter 1: Generative Effects/6 Galois connections/1 Definition and examples/Galois connection}{Galois connections}\,.
          \begin{itemize}
            \item Alternatively, start from definition of \href{doc/1 math/Seven Sketches in Compositionality/Chapter 2: Resource theories/5 Computing presented V-categories with matrix mult/1 Monoidal closed preorders/1 Closed SMP}{closed symmetric monoidal preorder} and substitute $v \multimap w$ for $a$, and note by reflexivity that  $(v \multimap w) \leq (v \multimap w)$
            \item Then we obtain $(v \multimap w) \otimes v \leq w$ (by symmetry of $\otimes$ we are done)
          \end{itemize}
    \item[d] Since $v \otimes I = v \leq v$, then \,\href{doc/1 math/Seven Sketches in Compositionality/Chapter 2: Resource theories/5 Computing presented V-categories with matrix mult/1 Monoidal closed preorders/1 Closed SMP}{the closed condition}\, means that $v \leq I \multimap v$
          \begin{itemize}
            \item For the other direction, we have $I \multimap v = I \otimes (I \multimap v) \leq v$ by (3)
          \end{itemize}
    \item[e] We need $u \otimes (u \multimap v) \otimes (v \multimap w) \leq w$
          \begin{itemize}
            \item This follows from two applications of \,the third part of this proposition\,.
          \end{itemize}
  \end{itemize}