\begin{itemize}
    \item Visual representations for building new relationships from old.
    \item For a preorder without a monoidal structure, we can only chain relationships linearly (due to transitivity).
    \item For a \href{doc/1 math/Seven Sketches in Compositionality/Chapter 2: Resource theories/2 Symmetric monoidal preorders/1 Definition and first examples/1 Symmetric monoidal structure on a preorder}{symmetric monoidal structure}, we can combine relationships in series and in parallel.
    \item Call boxes and wires \emph{icons}
    \item Any element $x \in X$ can be a label for a wire. Given \emph{x} and \emph{y}, we can write them as two wires in parallel or one wire $x \otimes y$; these are two ways of representing the same thing.
    \item Consider a wire labeled $I$ to be equivalent to the absence of a wire.
    \item Given a $\leq$ block, we say a wiring diagram is \emph{valid} if the monoidal product of elements on the left is less than those on the right.
    \item Let's consider the properties of the order structure:
          \begin{itemize}
            \item Reflexivity: a diagram consisting of just one wire is always valid.
            \item Transitivity: diagrams can be connected together if outputs = inputs
            \item Monotonicity: Stacking two valid boxes in parallel is still valid.
            \item Unitality: We need not worry about $I$ or blank space
            \item Associativity: Need not worry about building diagrams from top-to-bottom or vice-versa.
            \item Symmetry: A diaagram is valid even if its wires cross.
          \end{itemize}
    \item One may regard crossing wires as another \emph{icon} in the iconography.
    \item Wiring diagrams can be thought of as graphical proofs
          \begin{itemize}
            \item If subdiagrams are true, then the outer diagram is true.
          \end{itemize}
  \end{itemize}
