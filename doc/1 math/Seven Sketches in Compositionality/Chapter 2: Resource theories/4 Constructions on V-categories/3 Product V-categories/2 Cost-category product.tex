% TAG Example
\begin{itemize}
    \item Let $\mathcal{X}$ and $\mathcal{Y}$ be the \href{doc/1 math/Seven Sketches in Compositionality/Chapter 2: Resource theories/3 Enrichment/3 Lawvere metric spaces/2 Lawvere metric space}{Lawvere metric spaces} (i.e. \textbf{Cost}\href{doc/1 math/Seven Sketches in Compositionality/Chapter 2: Resource theories/3 Enrichment/1 V-categories/1 V-category}{categories}) defined by the following weighted graphs.
    \item \begin{tikzcd}\mathcal{X}:= & A \arrow[r, "2"]            & B \arrow[r, "3"]            & C \\\mathcal{Y}:= & P \arrow[r, "5", bend left] & Q \arrow[l, "8", bend left] &  \end{tikzcd}
    \item The product can be represented by the following graph: \,\begin{tikzcd}{(A,P)} \arrow[r, "2"] \arrow[d, "5"', bend right]  & {(B,P)} \arrow[r, "3"] \arrow[d, "5"', bend right]  & {(C,p)} \arrow[d, "5"', bend right] \\{(A,Q)} \arrow[r, "2"'] \arrow[u, "8"', bend right] & {(B,Q)} \arrow[r, "3"'] \arrow[u, "8"', bend right] & {(C,Q)} \arrow[u, "8"', bend right]\end{tikzcd}\,
    \item The distance between any two points $(x,y),(x',y')$ is given by \,the sum $d_X(x,x)+d_Y(y,y)$\,.

    \item We can also consider the \textbf{Cost}-\href{doc/1 math/Seven Sketches in Compositionality/Chapter 2: Resource theories/3 Enrichment/1 V-categories/1 V-category}{categories} as matrices

          \begin{minipage}{0.48\textwidth}
            \begin{tabular}{|l|l|l|l|}
              \hline
              $\mathcal{X}$ & A        & B        & C \\ \hline
              A             & 0        & 2        & 5 \\ \hline
              B             & $\infty$ & 0        & 3 \\ \hline
              C             & $\infty$ & $\infty$ & 0 \\ \hline
            \end{tabular}
          \end{minipage}

          \begin{minipage}{0.48\textwidth}

            \begin{tabular}{|l|l|l|}
              \hline
              $\mathcal{Y}$ & P & Q \\ \hline
              P             & 0 & 5 \\ \hline
              Q             & 8 & 0 \\ \hline
            \end{tabular}
          \end{minipage}

          \begin{minipage}{0.48\textwidth}
            \begin{tabular}{|l|l|l|l|l|l|l|}
              \hline
              $\mathcal{X}\times\mathcal{Y}$ & (A,P)    & (B,P)    & (C,P) & (A,Q)    & (B,Q)    & (C,Q) \\ \hline
              (A,P)                          & 0        & 2        & 5     & 5        & 7        & 10    \\ \hline
              (B,P)                          & $\infty$ & 0        & 3     & $\infty$ & 5        & 8     \\ \hline
              (C,P)                          & $\infty$ & $\infty$ & 0     & $\infty$ & $\infty$ & 5     \\ \hline
              (A,Q)                          & 8        & 10       & 13    & 0        & 2        & 5     \\ \hline
              (B,Q)                          & $\infty$ & 8        & 11    & $\infty$ & 0        & 3     \\ \hline
              (C,Q)                          & $\infty$ & $\infty$ & 8     & $\infty$ & $\infty$ & 0     \\ \hline
            \end{tabular}
          \end{minipage}

    \item    Can view this as a 2x2 grid of 3x3 blocks: each is \,a $\mathcal{X}$ matrix shifted by $\mathcal{Y}$\,.
  \end{itemize}