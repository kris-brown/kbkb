% TAG Exercise
\begin{itemize}
    \item Let $M$ be a set and $\mathcal{M}:=(P(M),\subseteq, M, \cap)$ be the \href{doc/1 math/Seven Sketches in Compositionality/Chapter 2: Resource theories/2 Symmetric monoidal preorders/1 Definition and first examples/1 Symmetric monoidal structure on a preorder}{symmetric monoidal preorder} whose elements are subsets of $M$.
    \item Someone says "for any set $M$, imagine it as the set of modes of transportation (e.g. car, boat, foot)". Then an $\mathcal{M}$ \href{doc/1 math/Seven Sketches in Compositionality/Chapter 2: Resource theories/3 Enrichment/1 V-categories/1 V-category}{category} $\mathcal{X}$ tells you all the modes that will get you from \emph{a} all the way to \emph{b}, for any two points $a,b \in Ob(\mathcal{X})$
          \begin{enumerate}
            \item Draw a graph with four vertices and five edges, labeled with a subset of $M=\{car,boat,foot\}$
            \item From this graph it is possible to construct an $\mathcal{M}$ \href{doc/1 math/Seven Sketches in Compositionality/Chapter 2: Resource theories/3 Enrichment/1 V-categories/1 V-category}{category} where the hom-object from \emph{x} to \emph{y} is the union of the sets for each path from \emph{x} to \emph{y}, where the set of a path is the intersection of the sets along the path. Write out the corresponding 4x4 matrix of hom-objects and convince yourself this is indeed an $\mathcal{M}$ \href{doc/1 math/Seven Sketches in Compositionality/Chapter 2: Resource theories/3 Enrichment/1 V-categories/1 V-category}{category}.
            \item Does the person's interpretation look right?
          \end{enumerate}
  \end{itemize}