\begin{itemize}
    \item Monoidal product acts on objects:
          \begin{itemize}
            \item $\mathcal{X} \times \mathcal{Y}((x,y),(x',y'))$ := $\mathcal{X}(x,x') \otimes \mathcal{Y}(y,y')$
          \end{itemize}
    \item Monoidal product acts on morphisms:
          \begin{itemize}
            \item $\phi \times \psi((x_1,y_1),(x_2,y_2))$ := $\phi(x_1,x_2)\otimes\psi(y_1,y_2)$
          \end{itemize}
    \item Monoidal unit is the $\mathcal{V}$ category $1$
    \item Duals in $\mathbf{Prof}_\mathcal{V}$ are just \href{doc/1 math/Seven Sketches in Compositionality/Chapter 2: Resource theories/4 Constructions on V-categories/2 Enriched functors/4 Exercise 2-73}{opposite categories}
          \begin{itemize}
            \item For every $\mathcal{V}$ category, $\mathcal{X}$, its dual is $\mathcal{X}^{op}$
            \item The unit and counit look like identities
                  \begin{itemize}
                    \item The unit is a $\mathcal{V}$ profunctor $1 \overset{\eta_\mathcal{X}}\nrightarrow \mathcal{X}^{op} \times \mathcal{X}$
                    \item Alternatively $1 \times \mathcal{X}^{op} \times \mathcal{X}\xrightarrow{\eta_\mathcal{X}}\mathcal{V}$
                    \item Defined by $\eta_\mathcal{X}(1,x,x'):=\mathcal{X}(x,x')$
                    \item Likewise for the co-unit: $\epsilon_\mathcal{X}(x,x',1):=\mathcal{X}(x,x')$
                  \end{itemize}
          \end{itemize}
  \end{itemize}
