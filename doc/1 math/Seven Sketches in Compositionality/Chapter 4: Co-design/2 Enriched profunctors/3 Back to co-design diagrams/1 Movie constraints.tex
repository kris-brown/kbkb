% TAG Example
\begin{itemize}
    \item Consider the \href{doc/1 math/Seven Sketches in Compositionality/Chapter 4: Co-design/2 Enriched profunctors/1 Feasibility relationships as Bool-profunctors/1 Feasilibiliy relation}{feasibility relation}
            \begin{tikzcd}
              T \arrow[rd] &                        &             \\
              & \boxed{\phi} \arrow[r] & Money \\
              E \arrow[ru] &                        &
            \end{tikzcd}
          \begin{itemize}

            \item $T:=\boxed{mean\rightarrow nice}$
            \item $E:=\boxed{boring\rightarrow funny}$
            \item $Money:=\boxed{100 K\rightarrow 500 K \rightarrow 1M}$
          \end{itemize}
    \item A possible \href{doc/1 math/Seven Sketches in Compositionality/Chapter 4: Co-design/2 Enriched profunctors/1 Feasibility relationships as Bool-profunctors/1 Feasilibiliy relation}{feasibility relation} is here:
            \begin{tikzcd}
              & {(nice,funny)}                                             &                                                &  & 1M             \\
              {(mean, funny)} \arrow[ru] \arrow[rrrru, blue, dotted] &                                                            & {(nice, boring)} \arrow[lu] \arrow[rr, blue, dotted] &  & 500K \arrow[u] \\
              & {(mean, boring)} \arrow[ru] \arrow[lu] \arrow[rrr, blue,dotted] &                                                &  & 100K \arrow[u]
            \end{tikzcd}

    \item This says, e.g., a nice but boring movie costs \, \$500,000 \, to produce.
    \item We can infer that we can also make a mean/boring movie with \,what we can produce nice/boring movie\,.

  \end{itemize}
