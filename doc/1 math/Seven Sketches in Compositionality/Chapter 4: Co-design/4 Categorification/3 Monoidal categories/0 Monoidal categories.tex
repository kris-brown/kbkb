\begin{itemize}
    \item Just like \href{doc/1 math/Seven Sketches in Compositionality/Chapter 1: Generative Effects/3 Preorders/1 Preorder}{preorders} are special kinds of \href{doc/1 math/Seven Sketches in Compositionality/Chapter 3: Databases/2 Categories/1 Free Categories/1 Category}{categories}, \href{doc/1 math/Seven Sketches in Compositionality/Chapter 2: Resource theories/2 Symmetric monoidal preorders/1 Definition and first examples/1 Symmetric monoidal structure on a preorder}{symmetric monoidal preorders} are special kinds of \href{doc/1 math/Seven Sketches in Compositionality/Chapter 4: Co-design/4 Categorification/3 Monoidal categories/1 SMC}{monoidal categories}.
    \item Just as we can consider $\mathcal{V}$ \href{doc/1 math/Seven Sketches in Compositionality/Chapter 2: Resource theories/3 Enrichment/1 V-categories/1 V-category}{categories} for a \href{doc/1 math/Seven Sketches in Compositionality/Chapter 2: Resource theories/2 Symmetric monoidal preorders/1 Definition and first examples/1 Symmetric monoidal structure on a preorder}{symmetric monoidal preorder}, we can consider $\mathcal{V}$ \href{doc/1 math/Seven Sketches in Compositionality/Chapter 2: Resource theories/3 Enrichment/1 V-categories/1 V-category}{categories} when $\mathcal{V}$ is a \href{doc/1 math/Seven Sketches in Compositionality/Chapter 4: Co-design/4 Categorification/3 Monoidal categories/1 SMC}{monoidal category}.
    \item One difference is that associativity is up to isomorphism: e.g. products in set $S \times (T \times U)$ vs $(S \times T) \times U$
    \item When the isomorphisms of a \href{doc/1 math/Seven Sketches in Compositionality/Chapter 4: Co-design/4 Categorification/3 Monoidal categories/1 SMC}{symmetric monoidal category} are replaced with equalities, we call it \emph{strict}
          \begin{itemize}
            \item Due to "Mac Lane's coherence theorem" we can basically treat all as strict...something we implicitly do when writing wiring diagrams.
          \end{itemize}

  \end{itemize}
