\begin{itemize}
    \item Because \textbf{Feas} is a \href{doc/1 math/Seven Sketches in Compositionality/Chapter 4: Co-design/3 Categories of profunctors/2 The categories V-Prof and Feas/1 V-profunctor category}{category}, we can describe \href{doc/1 math/Seven Sketches in Compositionality/Chapter 4: Co-design/2 Enriched profunctors/1 Feasibility relationships as Bool-profunctors/1 Feasilibiliy relation}{feasibility relation} using one-dimensional wiring diagrams:

          $\xrightarrow{a}\boxed{f}\xrightarrow{b}\boxed{g}\xrightarrow{c}\boxed{h}\xrightarrow{d}$
    \item We need more structure to talk about multi-input/output case.
    \item This chapter restricts discussion to skeletal quantales, but it is not hard to extend to non-skeltal ones. Two ways:
          \begin{enumerate}
            \item Let morphisms of $\mathbf{Prof}_\mathcal{V}$ be isomorphism classes of $\mathcal{V}$ profunctors. (analogous to trick used when defining \textbf{Cospan} category.)
            \item We could relax what we mean by category (requiring composition to be unital/associative `up to isomorphism' ... this is known as bicategory theory).
          \end{enumerate}
  \end{itemize}
