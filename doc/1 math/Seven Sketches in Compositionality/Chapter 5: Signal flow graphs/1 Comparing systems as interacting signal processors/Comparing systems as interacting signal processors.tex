\begin{itemize}
    \item Cyber-physical systems have tightly interacting physical and computational parts
    \item Model system behavior as the passing around and processing of signals (e.g. real numbers).
    \item Interaction can be thought of as variable sharing
          \begin{itemize}
            \item Physical coupling between train cars forces their velocity to be the same
          \end{itemize}
    \item Categorical treatment of \href{doc/1 math/Seven Sketches in Compositionality/Chapter 4: Co-design}{co-design problem} allowed us to evaluate composite feasibility relation from a diagram of simpler feasibility relations.
          \begin{itemize}
            \item Likewise we can evaluate whether two different signal flow graphs specify the same composite system
            \item (perhaps to validate that a system meets a given specification)
          \end{itemize}
    \item Symmetric monoidal preorders were a special class of symmetric monoidal category where \emph{morphisms} are constrained to be simple (at most one between any pair of objects).
          \begin{itemize}
            \item Signal flow diagrams can be modeled by props, which is a special class of symmetricf monoidal category where \emph{objects} are constrained to be simple (all are generaated by the monoidal product of a single object).
            \item In the former we have to label wires but not boxes. Vice-versa for props.
          \end{itemize}
  \end{itemize}
