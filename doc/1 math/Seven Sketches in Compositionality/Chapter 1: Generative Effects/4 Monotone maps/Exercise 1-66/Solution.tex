This is the \emph{Yoneda lemma for preorders} (up to equivalence, to know an element is the same as knowing its \href{doc/1 math/Seven Sketches in Compositionality/1 Chapter 1: Generative Effects/3 Preorders/Upper set}{upper set}).
  \begin{enumerate}
    \item \,This is basically the definition an upper set starting at some element.\,
    \item Interpreting the meaning of the preorder in the domain and codomain of $\uparrow$, this boils down to showing \,$p \leq p'$\, implies \,$\uparrow(p') \subseteq \uparrow(p)$\,
          - This is shown by noting that $p' \in \uparrow(p)$ and \,anything `above' $p'$ (i.e. $\uparrow(p')$) will therefore be in $\uparrow(p)$\,.
    \item Forward direction \,has been shown above\,
          - The other direction is shown just by noting that $p\prime$ must be an element of \,$\uparrow(p\prime)$ and by the subset relation also in $\uparrow(p')$,\, therefore $p \leq p'$.
    \item \, \begin{tikzcd}
      B & & C &  & \{B\} & & \{C\} \\
      & A \arrow[lu] \arrow[ru] & {} \arrow[rr, "\uparrow", dotted, bend left] &  & {}    & \{A, B, C\} \arrow[ru] \arrow[lu] &
    \end{tikzcd} \,
  \end{enumerate}