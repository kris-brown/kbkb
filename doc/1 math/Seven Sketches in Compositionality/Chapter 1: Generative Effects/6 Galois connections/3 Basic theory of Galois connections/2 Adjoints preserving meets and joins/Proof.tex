\begin{itemize}

    \item Left adjoints preserve \href{doc/1 math/Seven Sketches in Compositionality/Chapter 1: Generative Effects/5 Meets and joins/1 Definition and basic examples/Meet and join}{joins}
          \begin{itemize}
            \item let $j = \vee A \subseteq P$
            \item Given \,\emph{f} is \href{doc/1 math/Seven Sketches in Compositionality/Chapter 1: Generative Effects/4 Monotone maps/1 Monotone map}{monotone}, $\forall a \in A: f(a) \leq f(j)$\,, i.e. we have $f(a)$ as an upper bound for $f(A)$
            \item To show it is a \emph{least} upper bound, take some arbitrary other upper bound \emph{b} for $f(A)$ and show that $f(j) \leq b$
                  \begin{itemize}
                    \item Because $j$ is the least upper bound of $A$, we have \,$j \leq g(b)$\,
                    \item Using the \href{doc/1 math/Seven Sketches in Compositionality/Chapter 1: Generative Effects/6 Galois connections/1 Definition and examples/Galois connection}{Galois connection}, we have \,$f(j) \leq b$\, showing that $f(j)$ is the least upper bound of $f(A) \subseteq Q$.
                  \end{itemize}
          \end{itemize}
      \item Right adjoints preserving meets is dual to this.
  \end{itemize}