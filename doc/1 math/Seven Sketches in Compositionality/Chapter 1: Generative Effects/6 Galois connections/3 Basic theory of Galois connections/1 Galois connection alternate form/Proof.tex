\begin{itemize}
    \item Forward direction
          \begin{itemize}
            \item Take any $p \in P$ and let $q = f(p) \in Q$
                  \begin{itemize}
                    \item By reflexivity, we have in $Q$ that \,$f(p) \leq q$\,
                    \item By definition of \href{doc/1 math/Seven Sketches in Compositionality/Chapter 1: Generative Effects/6 Galois connections/1 Definition and examples/Galois connection}{Galois connection}, we then have \,$p \leq g(q)$\,, so (1) holds.
                  \end{itemize}
            \item Take any $q \in Q$ and let $p = g(q) \in P$
                  \begin{itemize}
                    \item By reflexivity, we have in $P$ that \,$p \leq g(q)$\,
                    \item By definition of \href{doc/1 math/Seven Sketches in Compositionality/Chapter 1: Generative Effects/6 Galois connections/1 Definition and examples/Galois connection}{Galois connection}, we then have \,$f(p) \leq q$\,, so (2) holds.
                  \end{itemize}
          \end{itemize}

    \item Reverse direction
          \begin{itemize}
            \item Want to show $f(p)\leq q \iff p \leq g(q)$
            \item Suppose $f(p) \leq q$
                  \begin{itemize}
                    \item Since \,\emph{g} is monotonic\,, $g(f(p)) \leq g(q)$
                    \item but, because \,(1)\,, $p \leq g(f(p))$, therefore $p \leq g(q)$
                  \end{itemize}
            \item Suppose $p \leq g(q)$
                  \begin{itemize}
                    \item Since \,\emph{f} is monotonic\,, $f(p) \leq f(g(q))$
                    \item but, because \,(2)\,, $f(g(q)) \leq q$, therefore $f(p) \leq q$\end{itemize}
          \end{itemize}
  \end{itemize}