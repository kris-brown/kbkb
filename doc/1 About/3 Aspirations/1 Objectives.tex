\begin{itemize}
\item A reader (with no particular technical background) ought be able to drop into a random page and start reading. Although there may be jargon / required context, by following appropriate internal links it should become clear what the jargon means / what context is required. Then, the reader ought backtrack precisely as far as needed to make sense of the original page. If any page doesn't meet this standard, then it's something for me to fix. \begin{itemize}
\item Often, notes will be jotted down here first, and weeks/months later I'll re-read them and do edits to make sense of them (only in that second step do they become intelligble).
\end{itemize}
\item When reading something new, what should I do to best retain the information I find useful? I could just simply use sheer memory/willpower and not take notes, but I find this doesn't work for me. It helps to take notes, but taking isolated notes (on paper / in a document) ends up with the notes being abandoned for a variety of reasons. Many of those reasons are addressed by embedding the notes into a personal wiki, where the content is connected to other notes via internal links.
\item For example, I once came across the story of \ref{tortoise|Achilles and the Tortoise|meta example} in the course of a lecture series, and the story seemed very profound. However, I couldn't explain the significance of the story later on to peers. What was needed to see was what ideas it was connected to (and then again for those, since the immediate connections are still quite abstract).
\end{itemize}