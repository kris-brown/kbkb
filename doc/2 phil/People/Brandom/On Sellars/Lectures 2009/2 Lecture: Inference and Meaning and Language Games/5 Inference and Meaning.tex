A main argument of \textit{Inference and Meaning}\cite{sellars1953inference} is that any language that makes essential use of non-logical, descriptive vocabulary must be understand as having that vocabulary standing in materially good (rather than just logically good) inferences.

``Concepts as Involving Laws and inconceivable without them'' is the title of an unintelligible essay by Sellars (but the title is the thesis and intelligible).

Sellars answers the first major question by claiming logical vocabulary (more specifically, alethic modal vocabulary, about what's necessary and what's possible) has the expressive job of making explicit the material proprieties of inference that articulate the content of non-logical concepts. Frege is more explicit about this (that you can use this to distinguish logical vocabulary) than Sellars.

Dan Dennett argues that we have to take animals as grasping modus ponens because they treat some inferences as good and others as bad (see \href{doc/2 phil/Phil Situations/Montaigne Dog}{Montaigne's dog}). You could make explicit the practical capacity the animal has using a statement of disjunctive syllogism, but Sellars would ask what is the surplus value of invoking that explicit expression? (Over simply describing what is the dog can do).

There have to be some practical moves you're just allowed to make without them having to take the form of explicit premises (see \href{doc/2 phil/Phil Problems/What the Tortoise Said to Achilles|importance}{Tortoise and Achilles}). Sellars touches upon this in Reflections on Language Games. He talks about free/auxillary positions that you're always allowed to occupy. We could have the auxillary position $\forall x, \psi(x)\vdash \phi(x)$ which would license us to move from a position $\psi(a)$ to a position $\phi(a)$, but we could also encode this with position for each possible move ($\psi(a)\vdash \phi(a)$, $\psi(b)\vdash\phi(b)$, ...). He ways that we could imagine replacing positions with moves, but it's not possible to imagine all moves being replaced with positions (`a game without moves is Hamlet without the Prince of Denmark').

Sellars is addressing tradition that wants some small set of explicit principles in accordance with which to reason. Any inference you think is good that isn't derivable from that small set of principles (e.g. modus ponens) is actually an infamy (has some suppressed premises). This is early analytic philosophy's embrace of the new logic. Sellar's contrary view (radical at the time) is that actually the reasoning could be completely in order, just with material proprieties of reasoning. You can still give/ask for reasons and mean that $p$, but what the logic does is give you meta-linguistic control to talk about what is a good inference and say that $p \vdash q$ is a good inference. Sellars doesn't extrapolate from this that logic is an optional superstructure in our lives - we need to be able to think and talk about the goodness of inferences.

Brandom: logic is the organ of semantic self-consciousness. The set of concepts that lets us bring our endorsement of some inferences as good/bad (this endorsement as something that reasons can be given or asked for) into the game of giving/asking for reasons.

Example: $A\vdash B$ where $A$ is ``she asked me to hand her the dish towel" and $B$ is ``I shall hand her the dish towel''. Traditional analytic philosophy will call this an infamy since it does not explicitly state how her request engages my motivational structure. Sellars would want to say that this invocation of the desire makes explicit the endorsement of $A \vdash B$ rather than referring to some item of the world.

Sellars complains about Carnap treating logical consequence as a syntactically definable relation between sentences. Just writing down the rules under a heading `rules' instead of `axioms' isn't making explicit the normative force they have (it leaves out the rulishness - that a rule is a rule for \emph{doing} something). This is a subtle point that doesn't matter for many purposes, but Sellars believes it's important if you want to understand what's going on with reasoning.

Potential counterargument against sellars: subjunctive conditionals are not making explicit proprieties of inference, but in fact are descriptions about possible worlds. To address this, we note there are separate issues. Firstly, there's the question about whether it's intelligible to have descriptive vocabulary in play in a context where there's no counterfactual reasoning. E.g. Hume believes he understands empirical facts perfectly well (the cat is on the mat) but not statements about what's possible and necessary. But Kant saw that this isn't intelligble - you need to make a distinction about what's possible with the cat and what's not (it's possible for the cat to not be on the mat, but not possible for it to be larger than the sun) or else there's nothing you could say about the conctent of the concept of `cat' that I've got (it would be just a label). The second issue is the codifiability of proprieties of material inference by logical vocabulary: whether a possible worlds analysis is incompatible with seeing subjunctive conditionals as making properties of inference explicit. Sellars would like to see a possible worlds analysis that matches up.

``There's an important difference between logical / modal / normative predicates on the one hand, and such predicates as `red' on the other.'' There's nothing to the formal except their role in reasoning, indeed, their role and make as meta linguistics sort of making explicit something about the ground level. For the latter, he wants to argue that these predicates too are meaningful insofar as their role in reasoning, but it's less obvious.

``Red is a quality''. This conveys the same information as the syntactical sentence ``\textit{Red} is a one place predicate.'' See \href{doc/2 phil/People/Sellars/Quotes/Modalities and norms}{quote}. What you're \emph{doing} in asserting that premise from which to reason (couched in modal vocabulary) is endorsing a principle in accordance with which to reason (couched in normative vocabulary).

We cannot completely identify modal and normative statements. When I say "copper melts at 1084 degrees" one makes a claim that is true even if ther were no reasoners (so it can't be a claim directly \emph{about} inferences being good). What it \emph{conveys} is about inferences, not what it \emph{says}. Likewise, I say ``The sun is shining'' while I convey ``I believe the sun is shining.''

It might help to make progress toward understanding the say/convey distinction (which Sellars admits he's not clear about) by distinguishing two flavors of inference:
\begin{enumerate}
\item semantic inference: good in virtue of the contents of the premises and the conclusion
\item pragmatic inference: good in virtue of what you're doing in asserting the premises or the conclusion.
 \begin{itemize}
 \item e.g. John says `your book is terrible' and I infer that he's mad at me
 \item Geech embedding distinction between the two: we look at whether we'd endorse ``My book is terrible, then John is mad at me". Because we wouldn't, we know the inference is pragmatic.

 \end{itemize}
\end{enumerate}