You can't understand what it is for somebody to be \emph{saying} (and therefore thinking something) apart from the way they're treated by some community. That's a \emph{sense dependence}: you can't understand the one without the other.

It doesn't follow that somebody who could do all of this wouldn't have thoughts until/unless they were treated as having them; that would be \emph{reference dependence}.

So just as an unconnected example, illustrating that distinction: suppose I defined `beautiful' as ``would cause pleasure in someone''. Now, then I've instituted a \emph{sense dependence} between beauty and that sense and pleasure; if you can't understand the concept of pleasure, you can't understand the concept of beauty, which is a response dependent dependently defined consequence of it.

But now we ask, would there still be beauty if there were no pleasure? Were there beautiful sunset sunsets before there were any people to feel pleasure? That would be the reference dependence question. We say sure because they \emph{would} have caused pleasure (if there were anyone there to feel it). And we can say, in a possible world in which there never were humans, it still could be that if there were, they would have responded to the sunsets with pleasure.

So, we could say there's a \emph{sense dependence} between these concepts, but there doesn't need to be a \emph{reference dependence} between doesn't mean you can't have the one without having the other it just means you can't understand what one of them \emph{is} without understanding the other.

So the claim would be that's the relation between the thoughts and our normative attitudes are social attitudes. It's not that the thoughts pop into existence at that point for them.