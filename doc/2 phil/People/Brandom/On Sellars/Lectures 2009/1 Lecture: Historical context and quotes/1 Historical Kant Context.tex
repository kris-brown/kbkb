Kant was not in favor within analytic philosophy when Sellars began. ``One cannot open the door enough for Kant to get through while being able to slam it shut before Hegel gets through.'' (He's too interesting of a reader, and Hegel was the great bad of Anglophone philosophy). Though it's also ironic since Kant is incredibly analytical and science-driven.

Four ideas of Kant that mattered to Sellars:
\begin{enumerate}
\item \textbf{Kant's normative turn}:
\begin{itemize}
\item His normative understanding of discursive (i.e. relating to concepts) practice. Kant saw that what distinguishes judgments/intentional behavior from habitual behavior is that there are things that the agents are in a distintive sense \emph{responsible} for. this point is shared by the later Wittgenstein. The \href{doc/phil/Phil Situations/Childrens Game}{puzzles} that Wittgenstein offers us (along the way to trying to dissolve the presuppositions that make it puzzling) center around the normative significance of beliefs/desires/intentions.

\item The difference between us and it is not an \emph{ontological} distintion but rather a \emph{deontological} distinction. Downstream of this is many of Kant's innovations. The minimum unit of awareness/experience is the judgment (you need concepts already for that). The subjective form of judgment (the ``I think", the emptiest form of judgment) mark of "who is responsible for the judgment" vs the subjective form of judgment the mark of what you've made yourself responsible to. In virtue of having your made yourself responsible to what you say you're talking about, is what make it that you're representing (it sets the standards of correctness).
\end{itemize}

\item \textbf{Turning Rousseau's definition of freedom into demarcating the normative}: Rousseau said ``Obedience to a law that one has laid oneself is freedom''. Kant turned this around to distinguish constraint by norms from constraint by power. Where natural things are bound by rules, we are bound by our conceptions of rules (i.e. to the extent we acknowledge them). Our normative status depend on our attitudes.

\item \textbf{Pure concepts of the understanding}: In addition to concepts whose principle expressive job is to describe/explain empirical goings-on, there are concepts whose principle expressive job it is to make explicit the framework that makes description possible. These are known \textit{a priori}. Framework-explicating concepts. This is Kant's response to Hume, for how we can understand the modal force of laws in virtue of their non-modal description. The answer is in the description framework itself. The fact that there are necessarily relations that concepts have among another makes description possible (a concept being contentful at all requires it to have some necessary relations to other concepts). What Sellars means by `ushering philosphy from its Humean phase to its Kantian phase' is putting categories front and center. Trying to \emph{describe} the modal structure of the world or describe the space of possible worlds is to try to assimilate modal language into descriptivism, rather than seeing them as playing a different expressive role (Sellars saw Kant as putting this other option on the table).
\end{enumerate}. Difference between Humean thinking and Kantian thinking: do you take this categorial status in some form (rather than it being descriptive) - `laws of nature are not super-facts - you are not describing the world'. It's a transposed rule of inference.

Another Kantian idea: the distinction between phenomena and noumena. Kant radicalized the distinction between primary and secondary qualities (properties that are truly there vs properties that are due to us). He challenges us to divide the labor, what features is the world responsible for vs are we responsible for (e.g. the fact our theories are expressed in German/English)? This distinction lives in Sellars as the difference between the world (in the narrow sense) and the world (in the wider sense ... e.g. including norms that are only accessible from a participant's perspective).
