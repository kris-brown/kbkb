Even our best understanding of actual observable empirical facts doesn't yield an understanding of rules relating or otherwise governing them.

We can get the facts, but not these non-logical, non-relations of ideas (inferential relations among them).

This as an epistemological point: how do we know such statements can be justified? But it's also a semantic point: given what you can \emph{mean}, on Hume's account of impressions and ideas, there's nothing you could mean other than constant conjunction, when talking about a non-logical relation among these things.

The facts don't settle which of the things that actually happened \emph{had to happen}, nor which of the things that didn't happen \emph{were still possible}, i.e. not ruled out by laws concerning what did happen.

So, Hume, and following him, Quine took it that epistemologically (but mostly semantically) fastidious philosophers faced a stark choice, either:
\begin{itemize}
\item show how to explain what's expressed by modal vocabulary in non modal terms
\item learn to live without it: Show how you can do what you need to do without talk about laws and possibility and necessity and so on.
\end{itemize}

Hume saw that there was nothing that you could be labeling (or in his sense, describing) by these statements of laws. This is why Hume woke Kant from his dogmatic slumbers.

This stems from a kind of \href{doc/phil/People/Brandom/On Sellars/2009/Lecture04/Semantic Nominalism}{semantic nominalism} that drew sharp lines around what you couldn't describe. Hume understood saying/thinking something \emph{as} describing something. It falls out of this that saying how things might have been (or how things had to be) isn't describing anything. These are the consequences of a theory of description (in the narrow sense) and a descriptivist theory of language. Kant looks at Newton's physics and determines that these conclusions are wrong, so there must be some non-descriptive thing we can do.