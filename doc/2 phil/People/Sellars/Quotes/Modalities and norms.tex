``The language of modalities is a transposed language of norms.''

(Related to \href{doc/2 phil/People/Sellars/Quotes/Describing the world without modality|related to}{this quote})

What is the connection between the (alethic) modal sense of \emph{must} and the
 normative sense of \emph{must}.

Carnap uses `transposed' to talk about an unquoted word used in a sentence as a
transposed mode of speech (about use of a quoted-word). E.g. ``Red is a
quality.'' vs `` `Red' is a one-place predicate.''
