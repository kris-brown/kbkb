Sellars first big idea: what was needed was a functional theory of concepts (especially alethic/normative modalities), which would make their role in reasoning, rather than their supposed origin and experience their primary feature. Sellars takes modal expressions to be inference licenses.


Jerry Fodor's theory of semantic content in terms of nomological locking: can't directly say anything about modality directly. Not alone: Dretske and other teleosemantic literature. Sellars wants to argue this program will never work.

An axial idea of \ref{kant|Kant|An axial idea of Kant}: the framework that makes description possible has features which we can express with words (words whose job is not to describe\footnote{in the narrow sense} anything, but rather to make explicit features of the framework within which we can describe things).

The framework is often characterized with laws. With alethic modal vocabulary on the object side, normative vocabulary on the subject side.

\ref{kant|Kant|What Kant is concerned with} is concerned with features that are necessary conditions of the possibility of applying descriptive contents. How are statements like that sensibly thought of as true or false? (where the home language game is ordinary descriptive language) Kant says yes, in a sense, but have to be careful. If these kinds of claims are knowledge, what kind of justification is involved in it? Can we think of them as expressing even a kind of empirical knowledge? (after all, we learned laws of nature empirically, it seems). We have to think about the relationship between the framework and what you can say, in the framework, and what you can say about the framework. Kant was the first one worried about all that stuff. Sellars wants to find a meta framework for talking about the relations between talk within that framework of description, and talk about that framework of description.

