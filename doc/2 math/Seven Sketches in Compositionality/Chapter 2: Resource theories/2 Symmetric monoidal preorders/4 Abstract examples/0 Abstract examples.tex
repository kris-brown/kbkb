\begin{itemize}
    \item Booleans important for the notion of enrichment.
          \begin{itemize}
            \item Enriching in a \ref{Symmetric monoidal structure on a preorder|symmetric monoidal preorder|referenced} $\mathcal{V}=(V,\leq,I,\otimes)$ means "letting $\mathcal{V}$ structure the question of getting from \emph{a} to \emph{b}"
            \item Consider $\mathbf{Bool}=(\mathbb{B},\leq,true,\land)$
                  \begin{itemize}
                    \item The fact that the underlying set is $\{false, true\}$ means that ``getting from \emph{a} to \emph{b} is a true/false question"
                    \item The fact that $true$ is the monoidal unit translates to the saying ``you can always get from \emph{a} to \emph{a}"
                    \item The fact that $\land$ is the moniodal product means ``if you can get from \emph{a} to \emph{b} \textbf{and} \emph{b} to \emph{c} then you can get from \emph{a} to \emph{c}"
                    \item The `if ... then ...' form of the previous sentence is coming from the order relation $\leq$.    \end{itemize}

          \end{itemize}
  \end{itemize}
