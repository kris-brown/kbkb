
\begin{itemize}
    \item The category of graphs as a \ref{Functor category|functor category|referenced}
    \item Schema for graphs: $\mathbf{Gr}:=\boxed{\overset{Arr}\bullet \overset{src}{\underset{tar}{\rightrightarrows}}\overset{Vert}\bullet}$
    \item A graph \ref{Database instance|instance|referenced} has a set of points and a set of arrows, each of which has a source and target.
    \item There is a bijection between graphs and \textbf{Gr} instances
    \item The objects of \textbf{GrInst} are graphs, the morphisms are \emph{graph homomorphisms} (\ref{Natural transformation|natural transformations|referenced} between two \textbf{Gr} to \textbf{Set} \ref{Functor|functors|referenced})
          \begin{itemize}
            \item Each graph homomorphism contains two components, which are morphisms in \textbf{Set}:
                  \begin{enumerate}
                    \item \,$G(Vert) \xrightarrow{\alpha_{vert}} H(vert)$\,
                    \item \,$G(Arr) \xrightarrow{\alpha_{arr}} H(Arr)$\,
                  \end{enumerate}
            \item Naturality conditions
                  \begin{enumerate}
                    \item \begin{tikzcd}G(Arr) \arrow[r, "\alpha_{Arr}"] \arrow[d, "G(src)"'] & H(Arr) \arrow[d, "H(src)"] \\G(Vert) \arrow[r, "\alpha_{Vert}"']                   & H(Vert)                   \end{tikzcd}
                    \item \begin{tikzcd}G(Arr) \arrow[r, "\alpha_{Arr}"] \arrow[d, "G(tar)"'] & H(Arr) \arrow[d, "H(tar)"] \\G(Vert) \arrow[r, "\alpha_{Vert}"']                   & H(Vert)                   \end{tikzcd}
                  \end{enumerate}
          \end{itemize}
  \end{itemize}
