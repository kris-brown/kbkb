Sellars has two good ideas associated with material inference:
\begin{enumerate}
\item There \emph{are} some inferences that are good, not in virtue of their logical form.
\item Turn the above thought on its head and say: we can understand the content of these descriptive terms in terms of the materially good inferences they appear in (as premises or conclusions).
\end{enumerate}

%Historical aside:
%This idea is connected to Bolzano (a contemporary of Frege's), who thought about how abstraction was connected to reasoning. We look at a good inference and the class of substitutions under which it remains good. This allows us to move from an equivalence relation to the equivalence class. Frege takes this and uses this as the basis for his logic and metaphysics. his notion of a function is understood in terms of substitutions / inter-substitutions.
%Quine picked up this substitution methodology in some of his technical writings (e.g. his essay on Carnap on logical truth). Quine talks about logical things as ones in which all non-logical vocabulray occurs vacuously (not in a way that's essential to the goodness of inference).

By this account, material proprieties of inference are more fundamental than / conceptually prior to logical validity. You have to start with the notion of a good inference in order to understand what a logically good inference is.
