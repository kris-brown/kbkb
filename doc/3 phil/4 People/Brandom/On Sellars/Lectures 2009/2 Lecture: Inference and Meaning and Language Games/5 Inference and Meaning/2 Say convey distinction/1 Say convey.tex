When I say "copper melts at 1084 degrees" one makes a claim that is true even if there were no reasoners (so it can't be a claim directly \emph{about} inferences being good). What it \emph{conveys} is about inferences, not what it \emph{says}. Likewise, I say ``The sun is shining'' while I convey ``I believe the sun is shining.''

It might help to make progress toward understanding the say/convey distinction (which Sellars admits he's not clear about) by distinguishing two flavors of inference:
\begin{enumerate}
\item semantic inference: good in virtue of the contents of the premises and the conclusion
\item pragmatic inference: good in virtue of what you're doing in asserting the premises or the conclusion.
 \begin{itemize}
 \item e.g. John says `your book is terrible' and I infer that he's mad at me
 \item Geech embedding distinction between the two: we look at whether we'd endorse ``My book is terrible, then John is mad at me". Because we wouldn't, we know the inference is pragmatic.
 \end{itemize}
\end{enumerate}