Potential counterargument against Sellars: subjunctive conditionals are not making explicit proprieties of inference, but in fact are descriptions about possible worlds. To address this, we note there are separate issues. Firstly, there's the question about whether it's intelligible to have descriptive vocabulary in play in a context where there's no counterfactual reasoning. E.g. Hume believes he understands empirical facts perfectly well (the cat is on the mat) but not statements about what's possible and necessary. But Kant saw that this isn't intelligble - you need to make a distinction about what's possible with the cat and what's not (it's possible for the cat to not be on the mat, but not possible for it to be larger than the sun) or else there's nothing you could say about the conctent of the concept of `cat' that I've got (it would be just a label). The second issue is the codifiability of proprieties of material inference by logical vocabulary: whether a possible worlds analysis is incompatible with seeing subjunctive conditionals as making properties of inference explicit. Sellars would like to see a possible worlds analysis that matches up.
