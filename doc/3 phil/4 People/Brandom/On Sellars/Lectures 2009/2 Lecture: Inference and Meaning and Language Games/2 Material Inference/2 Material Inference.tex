Let an \ref{material_conditional|inference|} be a declaration of the form $P \implies Q$
\begin{itemize}
    \item There, $P$ and $Q$ are \emph{logical} variables. We can also put other things in their place:
    \begin{itemize}
        \item Non-logical vocabulary, e.g. \emph{red}, \emph{cat}, or \emph{it's raining outside}
        \item Logical connectives: \emph{and}, \emph{or}, etc.
    \end{itemize}
\end{itemize}

We want to distinguish certain inferences as \emph{material inferences}, as distinct from logically-valid inferences.
\begin{itemize}

\item Logically-valid inferences:
    \begin{itemize}
        \item These are inferences that are true no matter what you plug in for the variables or substitute for the non-logical vocabulary.
        \item E.g. $(A \land {\rm it's\ raining}) \lor C \implies (C \lor {\rm it's\ raining})$
        \item This is true, regardless of what we substitute for $A$ and $C$ (or swap ``it's raining'' for anything, e.g. ``I own two cats'').
        \item \textbf{Descriptive terms appear \emph{vacuously}}
    \end{itemize}
\item Material inferences:
    \begin{itemize}
        \item These can be changed from a good material inference into a bad one by substituting some nonlogical vocabulary for different nonlogical vocabulary
        \item E.g. the material inference ``$a$ is red'' $\implies$ ``$a$ is colored'' will become false if we replace `colored' with `square'.
        \item \textbf{Descriptive terms appear \emph{essentially}}
    \end{itemize}
\end{itemize}
