\textcolor{red}{WARNING: Jotted down hastily, not yet cleaned up or fit for consumption.}

Sellars wants to give us a naturalistic account of intentionality.

Logical behaviorism / philosophical behaviorism
Def: the view that one can analyze without remainder intentional vocabulary / intentional concepts into purely behavior characterizations / dispositions to publicly observable behavior (specified in a non-intentional vocabulary).

Introduced in \emph{Empiricism in the Philosophy of Mind} (9 years earlier), distinct from what he here calls logical behaviorism. \emph{Logical behaviorism} refers to a view he attributes to Ryle. JB Watson and BF Skinner promoted this in psychology. Sellars never endorsed this because he saw this as being an application of \ref{instrumentalism|instrumentalism in the philosophy of science|Sellars}.

Observable things, at least we know they exist. Theoretical things, we've got to make risky inferences to get to them. But we can also make observational mistakes. Not just ``I thought it was a fox but it was a dog", but categorical observational mistakes. We can give some concept an observational role (e.g. declare that we can observe X's) yet no X's exist, i.e. no thing has such no thing with such circumstances of application and consequences of application. E.g. we can have a theory of acids. "Anything that's sour is an acid. And anything that's an acid will turn litmus paper red." Well, then we have observational access to acids. If we eventually find something that tastes sour that turns litmus paper blue, then it turns out there are no acids, even though we could observe them (or: had every reason to believe we could). Likewise: being a witch was observable (even though there are no such things).

The Plasticity of Mind is about bad theories incorporating observational practices, i.e. "What do you mean there are no K's. I can \emph{see} K's, there's one right there!"

So again, this is a response to someone saying we can distinguish theoretical from observable entities by pointing to the fact that we can make mistakes about whole categories of theoretical entities.


Just because our evidence for attributing mental states comes from behavior does not mean, unless you are an instrumentalist, that you have to be able to define intentional concepts in terms of behavior. (This doesn't mean that the intentional states are less real, just that we aren't in a position to observe anything but the behavior)
\begin{itemize}
\item  Digressions
\begin{enumerate}
\item \emph{Semantics} is a field with instrumentalist vs theoretical realist views. Michael Dummett is instrumentalist by observing the fact that meaning something is only understood through verbal behavior and concluding that any theory of meaning must be definable in terms of behavior. (A theoretical realist might postulate meanings as theoretical entities to explain verbal behavior and say our access to meanings is inferential and, if they are good theories, then verbal behavior gives us inferential access to something (meanings) that exist.)
\item MacDowell and Sellars agree (and disagree with almost all others) that what you hear when someone talks to you is the words themselves, rather than hearing noises and (by some inferential process) constructing the words. You have to actually actively do some work to hear that mere noises. This is evidenced by how difficult it was to tell computers how to recognize a smile in a picture. (Some say it's a contradiction to say that meanings are essentially normative yet, on the other hand, we sometimes can directly perceive them. But there's nothing in principle unobservable about normative states of affairs - see Sellars' criteria of observation below)
\end{enumerate}
\end{itemize}

(Controversial) Criteria for observation:
\begin{enumerate}
\item You have the capacity to reliably and differentially respond to some normative state of affairs
\item You have to have the concept and which is a matter of inferential articulation and practical mastery of inferential proprieties, involving it. And then if you can hook the one up to the other, you've turned what was beforehand a theoretical concept for you into into the concept of an observable
\end{enumerate}

Sellars wants to make sense of the notion of ``language as a rule-governed enterprise" (as essentially involving norms). Sellars believes that if your account language doesn't involven norms, you will be describing the vehicles by which we communicate, rather than what we're saying/meaning.

Reminder that, due to the \ref{regress_rules|regress argument|}, that we need to broaden our notion of `rule' from just explicit rules and need think of rules also as implicit in what we do. Sellars wants to better understand the relationship between implicit practical abilities and explicit representations of rules.

The question of whether meaning is a normative concept was brought to philosophical attention by Kripkenstein \cite{kripke1982wittgenstein}. In present literature, Hattiangadi and Katherin Gl{\"u}er have pushed back upon the idea that it is a normative concept, advanced by Brandom and MacDowell. Brandom feels it is because they haven't learned lessons from Sellars, in particular thinking of norms purely in terms of explicit presecriptions and not making the distinction between ought-to-be's and ought-to-do's.

``You can define \emph{possibility} in terms of \emph{not} and \emph{necessity}. You can define \emph{necessity} in terms of \emph{not} and \emph{possibility}. I think it's the beginning of wisdom to think of defining \emph{not} in terms of the relationship between \emph{possibility} and \emph{necessity}, but I'm the only human being who thinks that."

Grice on non-natural meaning: reduces what a linguistic expression $P$ means in terms of the meanings of thoughts and beliefs of those uttering $P$. Sellars isn't satisfied with this: the puzzling phenomena of meaning are common to both thought and language.

Sellars says ``ought-to-be's imply ought-to-do's" but is not exact about what quantifier: all or some? Brandom thinks `some' makes more sense, since there could be an ought-to-be requiring a state of affairs to change without telling us who has to do what to fix it (you need auxillary hypotheses to turn it into an ought-to-do). E.g. ``all clocks should be in sync''.

With a trainer, someone with concepts/rules can condition language learners to shape their behavior (teach them ought-to-be's). It's important that it's possible for the language enterprise get off the ground (i.e. without trainers). It's possible for some sort of selection process to naturally reinforce ought-to-be's (can be social but the conditioners need not be doing so intentionally).

We can deliberate making a distinction between ought-to-be's in the context of humans vs nonliving/nonsentient beings (e.g. ``plants ought to get enough water''). Ruth Millikan's work relevant. Connects to the Aristotelian account.

Consider ought to be's in the context of training animals: These rats ought to be in state $\phi$ whenever $\psi$.
\begin{itemize}
\item Could be just for rats, qua rats
 \begin{itemize}
 \item they ought to be eating when they're hungry, or something like that
 \end{itemize}
\item this could be something we want the rats to do
\begin{itemize}
    \item when they come to a branch in a maze, the rats go to the side that's painted blue and not to the side that's painted red.
    \item That's a regularity that ought to be not because we can read it off of the fundamental teleology of rats
    \item The conformity of the rats in question to this rule does \emph{not} require that they have a \emph{concept} $C$, e.g. of colors blue and red. We just require them to respond properly certain to differences emanating from $C$. This doesn't require even consciousness (photocells can respond differentially to colors).
\end{itemize}
\end{itemize}

``Recognitional capacity'' gets systematically used in two fundamentally different senses (an `accordion word')
\begin{itemize}
\item reliable, differential response
\item applying a concept
\end{itemize}

Important for Sellars that following an ought-to-be requires only the former sense.

We should talk about learning a language as `coming into the language' rather than `learning a language'. It's more like the way one comes into a city. You come to be able to take part in an ongoing practice, as opposed to getting some intellectual insight.

Teaching the very young child to say `purple' when showing her a purple lolipop is getting her to follow an ought-to-be just like the rat example. There is an ought-to-do for teachers of a language that they see to it that children produce the appropriate responses. This presupposes that the teachers do have a conceptual framework of `purple' and of `vocalize` and what it is for an action to be called by a circumstance. The learner is not required to have any of these concepts. The ought-to-be is explicit in the teacher's mind.




