\begin{table}[]
\begin{tabular}{|l|l|l|}
\hline
Name & Mr C &  Mr E\\ \hline
Stands for & constant conjunction & entailment \\ \hline
Represents & empiricists &  rationalists\\ \hline
Core of truth &  statements of necessary connection do not describe matter of factual states of affairs & what you're \emph{doing} when you make a modal claim is endorsing the propriety of a pattern of material inference \\ \hline
Core mistake & the only thing you can do with language is describe matter of factual states of affairs (therefore, laws must be descriptions of regularities) & statements of necessity describe entailment (still a descriptivist POV) \\ \hline
%&  &  \\ \hline
\end{tabular}
\caption{}
\label{tab:mrdebate}
\end{table}

Sellars take the dialectic through many turns instead of just saying what he thinks. He pretends to be even-handed until deciding to focus on tweaking Mr E.'s theory to make it work.

Need to distinguish four related types of claims:
\begin{enumerate}
\item The practical endorsement of infering that things are $B$'s from their being $A$'s. \begin{itemize}
\item This is presupposed by the act of describing (Kant-Sellars thesis)
\end{itemize}
\item The explicit statement that one may infer the applicability of $B$ from the applicability of $A$ \begin{itemize}
\item This can be asserted without understanding the expressions $A$ and $B$
\item I.e., it's syntactic; just a statement about the use of language
\item Someone who doesn't speak German can still say ``If $x$ is `rot', then $x$ is `farbig'."
\end{itemize}
\item The statement that $A$ physically entails $B$
\item The statement that $A$'s are necessarily $B$'s.
\end{enumerate}

Mr E was getting the \emph{content} of modal statements wrong; they aren't \emph{about} language.

That some inference is ok is something that is \emph{conveyed} by a modal claim, but it is not what is \emph{said}. (Analogy: John \emph{says/asserts/means} ``The weather is good today'', but John \emph{conveys} ``John thinks the weather is good today'' and John \emph{does not say} ``John thinks the weather is good today.''). Related to this \ref{modalities_norms|quote|referenced}.


What Sellars' conclusion ought to be: what one is \emph{doing} in making a modal assertion is endorsing a pattern of material inference. No need to take a stand on semantics. This is an expressivist view of modal vocabulary. Analogous to expressivism in ethics: what you are \emph{doing} in saying someone ought to do $X$ is endorsing doing $X$. We can try to understand the semantic/descriptive content\footnote{We cannot deny there is any descriptive content due to \ref{frege_geach|Frege-Geach argument|related to Sellars}} in terms of what one is \emph{doing} when we use the expressions.

