Let `possible world' \emph{mean} a physically possible world\footnote{not just metaphysically possible world, whatever that is}. This conceptual apparatus can be thought of  simply as a way of expressing what it is for two properties to be incompatible or to stand in a material consequential relation.

So we can express (in the language of possible worlds) the fact that it follows from (as a consequence) something's being copper that it melts at 1084 oC, for example.

David Lewis is presented as an example of misusing possible world semantics. He discards the connection between inference and the possible worlds. He takes the descriptivist position that all one can do is describe with language, but then says he is not an actualist (you can describe non-actual worlds in the same sense that you can describe the coin in your pocket as copper).

What is lost by merging description in the narrow sense and description in the wide sense? The connection to semantics. We've also gained an additional problem of justifying our claims to knowledge of the non-actual worlds.