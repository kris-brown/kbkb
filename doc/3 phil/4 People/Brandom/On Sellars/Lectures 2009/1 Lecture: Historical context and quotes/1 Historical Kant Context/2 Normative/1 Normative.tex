Kant's \emph{normative} understanding of discursive practice\footnote{i.e. practice relating to \emph{concepts}}.

\begin{itemize}
    \item How do we understand the difference between concept-using, sapient beings from mere responders to the natural environment? Here are two possible ways to think of it: \begin{itemize}
        \item $O$: An \emph{ontological} distintion: knowers are an actually different kind of thing (perhaps there is a presence of `mind stuff' or `spirit stuff').
        \item $D$: A \emph{deontological} distinction: we \emph{treat} knowers differently from objects. There are things that the agents are in a distinctive sense \emph{responsible} for \footnote{This point is shared by the later Wittgenstein. The \ref{childrens_game|puzzles|reference} that Wittgenstein offers us (along the way to trying to dissolve the presuppositions that make it puzzling) center around the normative significance of beliefs/desires/intentions.}.
    \end{itemize}
    \item Both sides treat $D$ as true, but Team $O$ furthermore believes $O$ is true and that the order of explanation is $O \implies D$. However, Team $D$ takes $D$ as essential and needs not make any claim about ontology.

    \item Downstream of this are many of Kant's innovations. \begin{itemize}
    \item The minimum unit of awareness/experience is the \emph{judgment} \begin{itemize}
    \item this comes from taking $D$ to be fundmental: it is the smallest thing we can be held responsible for
    \item Everything else (particular concepts like \emph{Fido the dog}, universal concepts like \emph{triangularity}, logical concepts) has to be understood in terms of the function it plays with respect to judgment.
    \end{itemize}
    \item The \emph{subjective} form of judgment (the ``I think..." that can accompany all judgments)
    \begin{itemize}
    \item Because it can accompany all representations, this is the emptiest form of judgment.
    \item The mark of "who is responsible for the judgment".
    \item To say ``\emph{I think} it is raining now.'' is to emphasize that \emph{I} am responsible (e.g. subject to criticism if you go outside and don't get wet).

    \end{itemize}
    \item The \emph{objective} form of judgment (the ``$x$ is ...'' or ``$x$ = ...'' for some object $x$). \begin{itemize}
    \item Mark of what you've made yourself responsible to.
    \item When saying ``That stone is 50 pounds.'', the stone has a certain authority over me (one looks to the stone to see whether I am right or wrong; it sets the standards of correctness). See the \ref{shopping|shopping list scenario|referenced}.
    \end{itemize}
    \end{itemize}
    \end{itemize}
