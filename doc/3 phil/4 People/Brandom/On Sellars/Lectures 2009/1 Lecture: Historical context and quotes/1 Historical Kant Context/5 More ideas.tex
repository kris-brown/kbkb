
A difference between Humean thinking and Kantian thinking: for Kant, laws of nature are not `super-facts' - they are not `describing the world'. Rather, they make explicit a rule of inference.

Another Kantian idea: the distinction between \emph{phenomena} and \emph{noumena}: Kant radicalized the distinction between primary qualities (properties that are truly there) and secondary qualities (properties that are due to us). He challenges us to divide the labor, what features is the world responsible for vs are we responsible for (e.g. the fact our theories are expressed in German/English)? This distinction lives in Sellars as the difference between the world \emph{in the narrow sense} and the world \emph{in the wider sense} (e.g. including norms that are only accessible from a participant's perspective).