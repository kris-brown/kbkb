
Difference between Humean thinking and Kantian thinking: do you take this categorial status in some form (rather than it being descriptive) - `laws of nature are not super-facts - you are not describing the world'. It's a transposed rule of inference.

Another Kantian idea: the distinction between phenomena and noumena. Kant radicalized the distinction between primary and secondary qualities (properties that are truly there vs properties that are due to us). He challenges us to divide the labor, what features is the world responsible for vs are we responsible for (e.g. the fact our theories are expressed in German/English)? This distinction lives in Sellars as the difference between the world (in the narrow sense) and the world (in the wider sense ... e.g. including norms that are only accessible from a participant's perspective).