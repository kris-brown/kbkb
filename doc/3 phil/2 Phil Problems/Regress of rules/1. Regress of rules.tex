\begin{itemize}
\item To follow a rule (e.g. `under circumstances $C$, do $A$', or `Whenever you see a cat, raise your hand'), we have to understand the concepts that are involved within (i.e. $C$ or \emph{cat}, or \emph{raising one's hand}).
\item However, to grasp a concept (such that the original rule can be followed) requires a further rule.
\item Put another way, because rules have many interpretations, for any rule $R$ we need another rule which tells us whether or not we correctly applied $R$.
\item If a law says `Every man must serve in the army', then it will naturally require a law for determining who qualifies as `man'. That law (say, `A man is whatever the scientific experts label with the word \emph{man}') will naturally require a law for determining who qualifies as `scientific expert'... the regress will continue if we try to adjudicate `scientific expert' with another rule.
\item This skepticism about rules is really a skepticism about a certain theory of \emph{meaning} (i.e. semantic skepticism).
\end{itemize}