We normally think that the content of judgments are dictated by the content of the concepts used inside of them. This feels especially right for artificial languages:

\begin{itemize}
\item Take ``If it's a $P$, then it's a $Q$.'' \begin{itemize}
\item Or, logically: $\forall x, P(x)\implies Q(x)$
\end{itemize}
\item The meaning of this statement seems to depend on what concepts $P$ and $Q$ we substitute in. E.g. with $P \mapsto {\rm red\ thing},\  Q \mapsto {\rm colored\ thing}$, it's a good a judgment, whereas $Q \mapsto {\rm rectangular\ thing}$ would no longer be a good judgment.
\end{itemize}

However, Kant turned this around (see \ref{kant_norm_turn|here|referenced}). In this view, judgments are prior (in the order of understanding) to concepts.

This is also related to a switch of priority made by Sellars: we normally think of logically-valid inferences (e.g. $A \land B \implies B \lor C$) as something we understand \emph{prior} to particular inferences (e.g. ``If it's red and triangular, then it's triangular or heavy''). Sellars calls these particular inferences \ref{material_inferences|material_inferences|mentioned} and argues that it is only through understanding them that we could understand logically-valid inferences.

One argument for our initial intuition is that the logically-valid inferences are \emph{a priori}, whereas the particular inferences are \emph{a posteriori}\footnote{The words \emph{priori} and \emph{posteriori} literally make the order clear.}. However, \ref{tortoise|Achilles and the Tortoise|relevant} feels relevant for arguing against this point of view: the logically-valid inferences exist \emph{a priori} as abstract mathematical/syntactical objects, but without any practical experience of actually making inferential moves, we don't have access to them \emph{qua} inferences.