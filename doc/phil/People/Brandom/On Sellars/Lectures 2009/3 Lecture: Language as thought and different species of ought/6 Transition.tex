Sellars story of how `the light dawns slowly over the whole'.

Both the infant and \href{https://en.wikipedia.org/wiki/Koko_(gorilla)}{Koko} the gorilla can be trained into a language (in the form of conforming to ought-to-be's). At some point the human makes a jump - they have the concept and can be a trainer of others. What's the nature of that jump?

For Sellars, this is a change in normative status, not a lightbulb that went off in one's head. Like the change on your 21st birthday, when suddenly doing the very same thing, making the same pen scratches that you could have made the day before, would \emph{not} be obliging yourself to pay the bank a certain amount of money every every month for the next 30 years. But after your 21st birthday, when you scratch your pen in exactly the same (physically descriptively, matter-of-factually) way, all of a sudden it has a hugely different normative significance because now you will be held responsible. You'll be taken to have undertaken commitment in a way in which you were not eligible to undertake that commitment by doing the very same thing descriptively, the day before.

When you get good enough at the language game moves, you do get acknowledged by the community. We don't characterize this physically-descriptively because we're not describing someone / some matter-of-factual boundary that has been crossed. We're not describing the child, we're placing the child in the space of reasons.

It's the difference between the one and a half year old, who toddles in to the living room. And as her first full sentence says, ``Daddy, the house is on fire." Well, one doesn't think that she has claimed that the house is on fire. She's managed to put these words together, this is good. If the four year old comes into the living room and says ``Daddy, the house was on fire", you hold her responsible, you say ``how do you know? Did you smell smoke? And you know, what should we be doing? What follows if the house is on fire? What should we be doing?" You take her to have claimed this to have undertaken a commitment and you hold her responsible for it. The difference is not some light that's going on. It's a difference in normative status, ultimately a difference in \href{doc/phil/People/Brandom/Slogans/Sapience}{social status}.

This is the difference between just conforming to the pattern, and actually making claims. The radically anti-Cartesian aspect of Sellars is that this is also the the difference between conforming to the pattern and \emph{having thoughts at all}.


However, as Dennett points out: you can treat any even inanimate object as an intentional system, e.g. this table as having the one desire that remain at the center of the universe. And the one belief that it is currently at the center of the universe, which is why it resists us moving it. (by extension, we treat our cats and dogs this way). So we should only treat things as thinking if we have to. Brandom takes an opposite view, that you should always treat something as talking if you can (note this is a \href{doc/phil/Phil Situations/Novel sentence|Used as example}{very high bar}).



The period prior to the child's mastery and social status as a language speaker has some peculiarities. His verbal behavior would express his thoughts but, to put it paradoxically, the child could not express them. The child isn't in a position to \emph{intentionally} say that things are thus-and-so, even though it is in a position to say that things are thus-and-so. So there's a question: which comes first, speaker's meaning or semantic meaning?

Semantic meaning is a matter of what the words mean. No agent involved in that. In English, the word `molybdenum' means the noble metal with 42 protons. Contrast with ``When Humpty Dumpty says `glory', he means \emph{a nice knockdown, drag-out fight}''. Grice says speaker meaning comes first. Sellars says that is a Cartesian way of thinking about things, that the primary meaning is what words mean in the language process.

If I claim the notebook is made out of copper, I have (whether I know it or not), committed it to melting at 1084 oC and that it conducts electricity. My words mean those things, whether or not I mean to.

The kid produces vocal (not yet verbal) noises until he is a member of the language community (his verbal noises conform to enough ought-to-be's).

As soon as he can say something, that's the expression of a thought. To take him to be saying is to be taking him to be thinking out loud. It's a further stage, when he can take expressing that thought as the object of an intention, and intentionally do as an action that say, before that, that's just an act, it's a performance, he can reliably produce appropriately, but not yet intentionally produce. An adult could be in this situation: \href{doc/phil/Phil Situations/Auction}{Auction example}. That's the sort of position that the kid (who's just crossed the line into being able to say something) is: she can produce a vocalization that will hold her responsible for, and which, accordingly, we take to express a thought. But she doesn't yet have the \emph{concept}. So, she can have the concept of its being red or the house being on fire. But not yet, the concept of \emph{endorsing something}, or of making a claim that he's saying can be a later development. And you need \emph{that} concept in order to \emph{intend} to be making a claim.

Important to make distinctions between different types of saying:
\begin{itemize}
    \item mere utterance (position of 1 year old)
    \item saying that things are thus-and-so
     \begin{itemize}
        \item having mastered the entries/exits/language-language moves, but no metalinguistic concepts
        \item Could be called ``merely thinking out loud''
        \item Can perform speech \emph{acts}.
        \item Can express that something is read or even a desire for something (``I'm taking that')
     \end{itemize}
    \item intentionally saying (\emph{telling} someone) that things are thus-and-so
    \begin{itemize}
        \item need concepts of asserting/believing as well as concepts of thus-and-so
        \item Can perform speech \emph{actions}.
        \item Self consciousness.
    \end{itemize}
\end{itemize}

We need to think of the child as being able to give evidence without the concept of evidence. This is important in the story of how the language game gets off the ground with the early hominids. But we have real experience with this: when teaching logic, it's helpful to teach students to have the practical mastery of writing proofs (prior to them having the concept of a proof). They first get familiar with the symbol pushing game. (proof is a strong form of evidence). This is very common in mathematics education.

