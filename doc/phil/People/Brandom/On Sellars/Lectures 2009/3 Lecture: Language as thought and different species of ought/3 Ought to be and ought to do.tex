% DEF

\begin{table}[]
    \begin{tabular}{|l|l|}
    \hline
  Ought to do's       &  Ought to be's \\ \hline
  Rule of action     & Rule of reflection \\ \hline
  If you're in circumstances $C$, do $A$ &  Pattern based judgment \\ \hline
  Conceptually articulated & Not necessarily conceptually articulated \\ \hline
  Rules of deliberation  & Rule of assessment/criticism \\ \hline
  First personal  &  Third personal judgment of some behavior \\ \hline
  What's appropriate for me to do?  & Given what you did, was it appropriate? \\ \hline
  The person subject to the rule is the one following the rule  &  There may be no particular agent at all \\ \hline
  Examples?  & \begin{itemize}
  \item ``One ought to feel sympathy for the bereaved''
  \item ``All clocks should strike midnight at the same time.''
  \item ``Plants ought to get enough water to flower''
  \end{itemize}   \\ \hline
    &  \\ \hline
    &  \\ \hline
    \end{tabular}
    \caption{}
    \label{tab:ought}
\end{table}

\begin{itemize}
\item A distinction fundamental for both `must' in the alethic and doxastic modal senses.

\item Sellars: You can't understand either of these kinds of oughts without understanding both. In particular, if you try to do everything with ought-to-do's:
 \begin{itemize}
 \item one would fall into a kind of Cartesianism: we'd need to think of linguistic episodes as essentially the sort of thing brought about by an agent whose conceptualizing is not linguistic.
 \item We'd be precluded from explaining what it means to have concepts in terms of the rules of the language. Ought to do's have the form of ``in circumstances $C$, do $A$'' - what language are $C$ and $A$ stated in? Regress of rules without ought-to-be's.
 \end{itemize}

\item This is important because natural way to think of rules is exlusively in terms of Ought to Do (Sellars himself advocated this earlier: ``A rule is always a rule for doing something"\cite{sellars1954some}).

\item There is also an analogous distinction involving permission, rather than obligation.

\end{itemize}



