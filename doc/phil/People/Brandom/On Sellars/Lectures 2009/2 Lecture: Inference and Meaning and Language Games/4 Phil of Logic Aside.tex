Aside: It took a while in the 20th century to realize that logic was not about
logical truth but rather about validity of inference. In classical logic can
you treat these interchangably, but not all (rough logics vs smooth logics -
whether the consequence relation can be determined by the set of all theorems).
Dummett has written about this issue.

What if we picked some other vocabulary (other than logical) to hold fixed? E.g.
substituting non-theological vocabulary for non-theological vocabulary.
``If justice is loved by the gods then justice is pious''. If no matter what we
substitute for justice the inference is good, we might say the sentence is true
in virtue of its theological form.

Philosophy of logic (See Quine's and Putnam's books both titled The Philosophy
of Logic) has two classic questions:
\begin{enumerate}
    \item a \emph{demarcation} question: what makes something logical vocabulary?
     \begin{itemize}
        \item Quine disallows second order quantifiers and the epilson of set
              theory, whereas Putnam allows them.
     \end{itemize}
    \item a \emph{correctness} question: which logical consequence relation to use:
      \begin{itemize}
      \item Classical? Intuitionistic? etc.
      \end{itemize}
\end{enumerate}

Sellars challenges this tradition (logical empiricism) by pointing out there is
a concern conceptually prior in the order of explanation to philosophy of
logic: materially good inferences.
