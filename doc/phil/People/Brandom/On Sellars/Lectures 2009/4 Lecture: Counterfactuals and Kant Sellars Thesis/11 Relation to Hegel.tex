Hegel says, ``By conceptual, I mean, what's articulated by relations of determinant negation\footnote{i.e. material incompatibility} and mediation\footnote{i.e. material consequence}".\footnote{Brandom says this is expressed, beginning in the section on Perception of \cite{hegel2007phenomenology}} He says, the objective world, as it is, independently of our activities, is conceptually articulated. It has a conceptual structure because he's a modal realist about it.

He thinks there are laws of nature. He thinks some things really follow from other things. For instance, that if a body with finite mass is accelerated, then a force was applied to it. he thinks that's a consequence, and the remaining relatively at rest, and having a force supplied you those are incompatible. Those are incompatible properties.

So he says, the objective world has a conceptual structure already that has nothing to do with our conceiving activity. We can see that as something else. Yes, our commitments can also stand in relations of material consequence and in compatibility, but the world, just as it comes, is already in conceptual shape.