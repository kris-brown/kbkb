Sellars wants to say it is the correctness of inferences connecting descriptive terms that make modal claims. Simultaneously, modal claims do not \emph{say} anything about inferences.

It's important that notion of \emph{saying} is wider than \emph{describing}. This is the denial of semantic descriptivism.

The pragmatic force associated with the modal claim is endorsing a pattern of inference.

What one says is that being a $B$ follows from being an $A$. This is not a statement \emph{about} inferences, it is a statement about a consequential relation. When one says ``$x$ being copper is incompatible with $x$ being an insulator", one is making a claim about the world (even if it is not describing it in the narrow sense). These are facts about what follows from what in the world. Given the auxillary hypothesis that our word `copper' means \emph{copper}, there are things we can say about inferences, but the fact that we need that auxillary hypothesis is proof that the statement itself isn't about inferences.

So modal claims are descriptive in the wide sense but not the narrow sense.



