Sellars wants to say it is the correctness of inferences connecting descriptive terms that make modal claims. Simultaneously, modal claims do not \emph{say} anything about inferences.

It's important that notion of \emph{saying} is wider than \emph{describing}. This is the denial of semantic descriptivism.

The pragmatic force associated with the modal claim is endorsing a pattern of inference.

What one says is that being a $B$ follows from being an $A$. This is not a statement \emph{about} inferences, it is a statement about a consequential relation. When one says ``$x$ being copper is incompatible with $x$ being an insulator", one is making a claim about the world (even if it is not describing it in the narrow sense). These are facts about what follows from what in the world. Given the auxillary hypothesis that our word `copper' means \emph{copper}, there are things we can say about inferences, but the fact that we need that auxillary hypothesis is proof that the statement itself isn't about inferences.

So modal claims are descriptive in the wide sense but not the narrow sense.


There are some serious concerns, though. Consider "There exist causal connections which have not yet been discovered". This is analogous to accepting the early emotivist line in ethics (thinking `ought' is a perfectly good concept, though not a descriptive one ... such that `Everybody ought to keep promises' contextually implies a wish, on the speaker's part, that promise keeping were a universal practice), and was then confronted with such statements as ``There are obligations which have not yet been recognized'' and ``Some of the things we think of as obligations are not obligations''

Quote: ``It is therefore important to realize that the presence in the object language of the causal modalities (and of the logical modalities and of the deontic modalities) serves not only to express existing commitments, but also to provide the framework for the thinking by which {303} we reason our way (in a manner appropriate to the specific subject matter) into the making of new commitments and the abandoning of old. And since this framework essentially involves quantification over predicate variables, puzzles over the `existence of abstract entities' are almost as responsible for the prevalence in the empiricist tradition of `nothing-but-ism' in its various forms (emotivism, philosophical behaviorism, phenomenalism) as its tendency to assimilate all discourse to describing.''

If we are to take causal modalities, seriously / at face value, we're going to have to worry about what abstract objects and what properties are.

Brandom thinks Sellars could have a simpler/more satisfying conclusion to the essay, but Sellars' nominalism (denial of existence for abstract objects/properties) prevents him from doing so.

In Brandom's view, material inferences are \emph{not} monotonic. The job of some scientific languages is to find concepts where we can state monotonic consequence relations. We can do that in fundamental physics, but hardly ever in the special sciences.

