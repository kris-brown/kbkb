\href{doc/phil/People/Brandom/On Sellars/Lectures 2009/4 Lecture: Counterfactuals and Kant Sellars Thesis/6 Semantic Nominalism}{Semantic nominalism} was universally held until Kant. (philosophy today hasn't yet graduated from its Humean to its Kantian phase)

If you could argue that standing in counterfactually, robust inferential relations to other descriptive terms, was an essential feature of the descriptive content of a concept (and you could argue that modal vocabularies had the expressive job of making those explicit), then you'd be in a position to argue for the Kant-Sellars thesis about modality. That would be to say that the expressive job of modal vocabulary is to make explicit the inferential relations between descriptive concepts (these are invisible to the empiricist).

Fork in road: Hume+Quine, or Kant+Sellars.

Labels are not descriptions. There's more to describing than labeling.

Consider mere labels. Elements on a tray have red or blue dots. Have they been described? If so, what have they been described as? If we add things to the tray, we don't know whether they deserve red or blue labels. At the very least, descriptions need a practice for applying to new cases. We can throw that in but still not have a description.

Suppose I'm trying to give you the concept of gleeb. I give you an infalliable gleebness tester. Do you have a description?

Sellars: ``It's only because the expressions in terms of which we describe objects, locate those objects in a space of implications that they describe it all rather than merely label.''

This is what a mere classifier have.

What's the difference between me and a parrot who has been trained to say `red' when presented a red object. The parrot isn't describing, but I am because my noise is situated in a space of implications: something follows for me from classifying that thing is red (that it's colored that it's spatially extended, that if it's a Macintosh Apple, it's right). And furthermore, other things can be evidence for the claim that it Scarlet is evidence for it or that it's a right Macintosh, Apple is evidence for it. And it excludes other classifications. That monochromatic patch is not green, if it's red, and so on.

The classifier focuses only on the \emph{circumstances} of application, not the \emph{consequences} of application. A way to answer what the red and blue dots describe an element \emph{as} is to say what follows from something having a blue or red dot (e.g. things labelled red are to be discarded). Now we have some descriptive content associated with the label.

Verificationism / classical american pragmatism focused purely on the \emph{consequences} of application.

To say we need both circumstances and consequences is to say that the inferences the concept plays into is an essential part of a concept.

If the only inferences we could make were truth functional relations, then a gleeb detector like thing could be sufficient to capture the concept of gleeb fully; however, we care about counterfactually robust inferences.

You can't count as understanding (i.e. grasp the meaning of) any descriptive expression / concept, unless you distinguish at least some of the inferences that it's involved (i.e. some of the connections within that space of implications) as counterfactually robust (i.e. ones that would still obtain, even if something that is true wasn't, or something that isn't true was). The claim is not that there are particular counterfactual inferences you need, but that you need at least \emph{some} to have the concept.

Examples:
\begin{itemize}
\item Chestnut trees produce chestnuts
\begin{itemize}
\item Unless they're immature / blighted
\item Whether or not it's raining on them now wouldn't affect the fact that chestnut trees produce chestnuts
\end{itemize}
\item Dry, well-made matches light if you strike them
\begin{itemize}
\item Not if there's not enough oxygen
\item The position of a distant beetle on a tree that doesn't affect whether this match lights
\end{itemize}
\item The hungry lioness would chase the nearby Gazelle
\begin{itemize}
\item Not if it were struck by lightning
\item But it would, whether or not the hyena were watching it
\end{itemize}
\end{itemize}

Q: do we have to have overlap in our distinctions we made to communicate about the same concept?
A: The conceptual content itself is a norm that settles which inferences are correct and which are incorrect. Then, you and I may have different views about where that line is drawn. And what makes it possible for us to communicate (to agree or to disagree) is that we've bound ourselves by the same norm by using the same word. You may think the melting point of copper is different from what I think it is. But we can still be disagreeing about copper because there's a fact of the matter about what you're committed to, on that issue, when you use the word `copper'.

Q: Is this a counter-example? You are asked to bring a thing to your lab (it just landed from an alien planet - you can't make any inferences about it).
A: It's complicated. Firstly, `thing' or `object' is not a sortal - you can't count them (they're pro-sortals, placeholders for sortals). If we supply one (e.g. place-occupying piece of mass - which would suffice give us some counterfactually robust inferences). If you don't supply one, you haven't thought about it. Related to Wittgenstein's \href{doc/phil/Phil Situations/Plate pointing}{plate example}.

\href{doc/phil/People/Kant}{Kant} says concepts are rules for judgment we \href{doc/phil/People/Brandom/Slogans/Concepts}{bind ourselves by}. That doesn't settle the question of how much of the law we need to know in order to bind ourselves by it. I don't have to know much about molybdenum to refer to molybdenum.



