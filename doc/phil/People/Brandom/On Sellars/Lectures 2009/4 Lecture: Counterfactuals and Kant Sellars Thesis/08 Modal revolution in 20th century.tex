Revolution in Anglophone philosophy, taking three phases:

\begin{enumerate}
    \item Kripke's possible-worlds semantics for modal logic (algebraic properties accessibility relation determine which modal system we're in) \begin{enumerate}
        \item Quine was a skeptic of modal talk, said that we didn't know what we really meant when talking about it.
        \item His concerns became unpopular with this development, as Kripke's semantics gave people the impression that we had a grasp on what we meant (in fact, alethic modal talk became the philosophical ground level for explaining other puzzling things (such as intentionality)).
        \item  Quine thinks his objection is not yet satisfied; of course we can explain modal concepts in terms of other modal concepts - what do we mean by \emph{possible world}? What do we mean by \emph{accessibility}?
    \end{enumerate}
    \item The previous development is only for modal logic, but Montague, David Kaplan, David Lewis, Stalnaker, etc. extended it to work for non-logical semantics as well. For example: \begin{itemize}
        \item In Lewis' \textit{General Semantics} \cite{lewis1976general}, if we decide to treat objects as semantic interpretant of our names, sets of possible worlds as the semantic interpretant of our sentences, then one-place predicates correspond to functions from objects to sets of possible worlds.
        \item `walks' is a function from objects to the worlds in which that object is walking.
        \item adverbs, like `slowly', is a function from one-place predicates to one-place predicates.
        \item This apparatus gives us a way to talk precisely about certain distinctions that come up in philosophy language. E.g. adverbs come in two flavors: \begin{itemize}
            \item Attributive: `walked \emph{slowly}' which implies one walked at all
            \item Non-attributive: `walked \emph{in one's imagination}'
    \end{itemize}
    \end{itemize}
    \item Kripke's \textit{Naming and Necessity} \cite{kripke1972naming} \begin{itemize}
    \item Introduced us to contingent \textit{a priori} and metaphysical necessity
    \end{itemize}
\end{enumerate}

Brandom says: the second is the most important, but Quine's objection was not addressed by Kripke's semantics nor the fact that it's useful in semantics to be able to use Kripke's apparatus. The reason why we should have gained comfortablity with alethic modal language is actually the \href{doc/phil/People/Brandom/On Sellars/Lectures 2009/Lecture: Counterfactuals and Kant Sellars Thesis/Kant Sellars thesis}{Kant-Sellars thesis}, which dispels empiricist worries of modal concepts being unintelligible.

But hardly anyone knows about that argument. If they did, it would color our current focuses and interests. For the Kant-Sellars thesis pertains to causal/physical modalities. But nowadays, the center of philosophical thought worries about logical modalities and metaphysical modalities - causal modality is boring to them. ``We're distracted by the shiny, new playground that Kripke offered us and lost sight of the modality that's philosophically most significant."