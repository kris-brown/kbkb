Semantic paradigm is the ``name-bearer'' relationship. E.g. the `Fido'-Fido relation, between the name `Fido' and the dog, Fido. Predicates/properties are just names that stick to the set.

If we stick together labels, we get descriptions (not a primitive name bearer relation, but one we can understand in terms of name-bearer relations). That's what language lets us do. Describe/classify things (as falling under languages).

Sellars calls this \emph{descriptivism}: what you do with language is describe things.

You find this not only in anglophone tradition but also in a pure form in Hussurl / semiotics. Derrida rejects Hussurl because there are some phenomena that can't be related by sign-signified relations, but he addresses it by saying it's all signs.

Kant found sentences special: you don't describe things with them, you say things with them. Theory of judgment. The Tractatus has no room for statements of natural law. For normative statements, it retreats into mysticism.

Semantic nominalism is atomistic - the relation of a name and its bearer doesn't turn on anything else.

