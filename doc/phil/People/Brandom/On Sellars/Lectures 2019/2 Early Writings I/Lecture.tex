
Three historical currents:

England: Bertrand Russell and GE Moore. Introducing analytic philosophy against absolute idealism (championed by Bradley, who was influenced by Hegel-inspired T.H. Greene who was reacting negatively to dominant British empiricists). Sellars admired On Denoting.

American: At turn of the century, German idealism dominated (Josiah Royce more popular speaker than Williams James). Two movements recoiling against this: American pragmatism and "new realism/critical realism" (AKA trinonminalists, including Sellars' father, which lost to pragmatism and logical empiricism).

German/Austrian: Marbourg (natural science focus of Kant) -> Carnap + Vienna Circle. Three periods, Aufbau (radical reductive empiricism: all statements must be definable in terms of immediate experience + logical vocabulary). Carnap then lightens up (in response to CI Lewis) and says statements must at least be able to be supported by evidence that comes from experience (replacing biconditional EXPERIENCE<->THEORETICAL with just a conditional EXPERIENCE->THEORETICAL;' there's a surplus on the theoretical side).
In parallel, Frankfurt school (Adorono/Walter Benjamin/Habermas), concerned with culture (and with Marxist inflection).

Tools of the syntactical phase of logical empiricism not adequate to address all general philosophical problems - it was improved by the semantic dimensions. He wants to turn the crank again to add a pragmatic dimension.

What did Sellars see of value in the reductivist Carnap? Carnap quote: ``A symbol is introduced (or, if it is already in use, is subsequently legitimized) by determining under what conditions it is to be employed in the representation of a state of affairs. The introduction or legitimization of the word 'horse', for instance, comes about by determining the conditions which must hold if we're to call something a horse, hence through statement of the distinguishing features of a horse or the definition of horse. We say of the symbol (that has been introduced / legitimized in such a way that we think is at least capable of legitimization) that it designates a concept. So, the symbol of a concept is a rule-governed symbol. Whether it be defined or not. Its use should above all be rule-governed. The symbol should not be employed in any old arbitrary way, but rather, in a determinate consistent way. Uniformity in the mode of employemnet can be secured either by explicitly laying down rules or merely through constant habit, linguistic usage. We have not yet said anything about what a concept is, but only for what it is for a symbol to designated a concept and this is all that can be said with any precision. But it's also enough, for when talk of concepts is meaningful, it invariably addresses concepts designated by symbols or concepts that can in principle be so designated. And such talk is basically alaways about these symbols and the laws of their use. The formation of a concept consists in the establishment of the law concerning the use of the symbol it is a word in the representation in a state of affairs''. Link to sellars quote: ``Grasp of a concept is always mastery of the use of a word" Even in the Aufbau, Carnap thinks of rule-governedness of symbols being crucial to the meaning of concepts.

One way to understand the core program of analytical philosophy: the project of elaborating the meanings of a puzzling vocabulary in terms of a base vocabulary (unproblematic) with logical vocabulary. Naturalism (trying to describe intensions/norms in terms of natural science)/empiricism (trying to describe laws in terms of sense experience) as an example. (Kris: is ordinary language philosophy an example?) If it's not possible to elaborate, then the vocabulary is seens as defective in some way.

Brandom tries to synthesize this view with pragmatism as exemplified by Wittgenstein/Sellars.