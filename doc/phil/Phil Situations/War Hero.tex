A man serves in a WWII. In a battle, he does something extremely heroicly, and lots of people see. He then gets knocked down, loses his dog tags, rolls down a hill. He wakes up later and has no memory of who he was until that point in his life. He wanders off and is eventually found by another regiment who has no idea who he is. He winds up in a hospital and makes his way back to America to go to college. Meanwhile, his original troop assumes the heroic soldier died (was probably blown to smithereens, since no trace of him other than dog tags) and he is awarded a posthumeous medal of honor. The solder gets interested in history and does a phd. He decides to writes about the battle he knows he was involved in (in some way). He's intruiged by the story of The War Hero and makes that his focus. Soon, he knows everything there is to know about The War Hero in his life up until the battle. \cite{castaneda1968logic}

One might say, casually, he knows more about The War Hero than The War Hero himself knew about himself.

Commentary by John Perry (Stanford):

This person has two distinct kinds of self knowledge:

\begin{itemize}
    \item Knowledge of the person one happens to be
    \begin{itemize}
        \item This is the normal kind of self-knowledge
        \item ``I am a graduate student in Berkekey"
        \item ``I have amnesia and no memory of anything before the battle"
    \end{itemize}
    \item Self-knowledge
    \begin{itemize}
        \item ``The War Hero was born in Cinincatti on a cold day."
        \item ``The War Hero was forced to wear shorts even during the winter.''
        \begin{itemize}
            \item If it's true, why is it true? Judging by the truth conditions of the sentence, it's true because a certain person born in Cininitati and was fordced to wear short pants, etc.
        \end{itemize}
    \end{itemize}
\end{itemize}

Most of us have both types of knowledge, but because they so easily run together we conflat the two.

