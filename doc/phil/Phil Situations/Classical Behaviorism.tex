This is related to the problem of: suppose you follow a rule. You use a
representation of that rule to train the next generation to follow the rule.
How do we know that the same rule is being passed on? Isn't it like a game of
telephone, given the ambiguity of following rules (gerrymandering problem - you
have a rule in your head and punish me for lifting the glass of water, but I
interpret this as punishment for lifting my arm or one of a million other
possible explanations). Brandom's response:

Well, maybe so and maybe it a given rule isn't as stable as we think. I mean,
it's only if it had enough coherence and enough stability, that we're here.
(Anthropomorphic principle). And when Wittgenstein harps on this, you know, unless
as a matter of fact, we tended to go on the same way when trained the same way,
to a remarkable extent, we wouldn't get a language game off the ground.

I should mention that this gerrymandering issue is what was wrong with classical behaviorism from an empirical point of
view. \footnote{What was wrong with
classical behaviorism, from a \emph{conceptual} point of view, is we can see it with
the wisdom of hindsight is just a larval stage on the way to functionalism. As
all of the considerations that lead people to think have direct stimulus
response connections, are satisfied still, if you allow intervening states. It's still an empirical undertaking, and so on. But there's a
lot more formal power, you can get Turing machines, if you can get functional
states, so you can get a lot farther. That's why
nobody should be a classical behaviorist anymore: be a functionalist, you get
all the advantages, and a lot more expressive power.} Remember, the stimulus and response were supposed to be
objective features of the critters you were looking at. So that the behavioral
scientist modeled on the natural scientist, her own conceptual scheme was not
supposed to be involved in characterizing the behavior of these critters. But
if you ask sort of classical studies, so I take the rat, and set him down four
steps away from the bar, and train him, then if he walks four steps forward and
presses down on the bar, he'll get a rat yummy. And that stimulus, let's say,
the light goes on, walks, four steps, pushes down on the bar gets a rat yummy.
We indoctrinate him with that, conditioned learning, he can do that. And now
we ask the behavioral scientist. And now if I put in eight steps away from
the bar, what do you predict he's going to do? Is he going to go four steps
forward and move his paw up and down? Is that the behavior that has been
associated with the stimulus? Or would you predict that he'll go eight steps
forward and press down on the bar. That is, the right description is that he'll
go from where he is to the bar and press on the bar? Well, the minute you think
about this, you realize that we can gerrymander, what he was taught,
there are many descriptions, that that are available to us for what he was
taught. And in fact, no one who works with the animals would expect him to
move four steps forward, and not be pressing on the bar. But why
is that? Is that something that you without importing any understanding of
this are objectively reading off of the situation? Or have you, in fact, all
along been importing, your characterization of what the regularity
is that you are, that you're characterizing? This is actually empirical as well a methodological
problem. What is the prediction that you're supposed to make at this point?
And how do you justify the one rather than rather than the other by your
methodological lights?
