\documentclass[12pt,a4paper]{report}
\usepackage{tikz}
\usepackage{hyperref}
\usepackage{amsmath}
\usepackage{amssymb}
\usepackage{amsthm}
\usetikzlibrary{arrows,positioning}
\tikzset{ >=stealth', punkt/.style={ rectangle, rounded corners, draw=black, very thick, text width=6.5em,minimum height=2em, text centered}, pil/.style={ ->, thick, shorten <=2pt, shorten >=2pt,}}
\begin{document}

 \href{doc/phil/PhilProblems.html}{Up} 
 \href{doc/phil/PhilProblems/BradleysProblem.html}{Next} 
 \href{doc/phil/PhilProblems/AgrippanTrilemma.pdf}{PDF} 
\title{Agrippan Trilemma}
There are only three ways of completing a proof:
\begin{itemize}
    \item The \emph{circular} argument, in which the proof of some proposition presupposes the truth of that very proposition
    \item The \emph{regressive} argument, in which each proof requires a further proof, ad infinitum
    \item The \emph{dogmatic} argument, which rests on accepted precepts which are merely asserted rather than defended
\end{itemize}

This is a dilemma for classical logic in particular; it could be addressed by introducing nonmonotonic logic (default and challenge). Some claims (e.g. first person observations) come with a default justification (which is not based on the justification of other claims). Yet it must be defended when challenged.
\end{document}