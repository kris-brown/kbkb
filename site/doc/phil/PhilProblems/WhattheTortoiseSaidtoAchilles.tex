\documentclass[12pt,a4paper]{report}
\usepackage{tikz}
\usepackage{hyperref}
\usepackage{amsmath}
\usepackage{amssymb}
\usepackage{amsthm}
\usetikzlibrary{arrows,positioning}
\tikzset{ >=stealth', punkt/.style={ rectangle, rounded corners, draw=black, very thick, text width=6.5em,minimum height=2em, text centered}, pil/.style={ ->, thick, shorten <=2pt, shorten >=2pt,}}
\begin{document}
 \href{doc/phil/PhilProblems/Universals.html}{Previous} 
 \href{doc/phil/PhilProblems.html}{Up} 

 \href{doc/phil/PhilProblems/WhattheTortoiseSaidtoAchilles.pdf}{PDF} 
\title{What the Tortoise Said to Achilles}

Lewis Caroll (1895)

Tortoise assumes $p$ and the conditional $p \implies q$, and Achilles tries to convince the Tortoise to accept $q$. He says that logic \emph{obliges} you to acknowledge $q$ in this case. Tortoise is willing to go along with this but demands that this rule be made explicit, so Achilles add an extra axiom $p \land (p \implies q) \implies q$. Achilles says that, now, you \emph{really} have to accept $q$ that we've written down $p$, $p \implies q$ and $p \land (p \implies q) \implies q$. But the Tortoise notes that, if that's really something logic obliges me to do, then it bears writing down ($p \land (p \implies q) \land (p \land (p \implies q) \implies q)$). Ad infinitum.

The most influential pragmatist work in the philosophy of logic, according to Brandom. The lesson: in any particular case, you can substitute a rule (that tells you you can go from this to that) with an axiom. But there have got to be some moves you can make without having to explicitly license them by a principle.  You've got to distinguish between 1.) premises from which to reason and 2.) principles in accordance with which to reason. This teaches an un-get-over-able lesson about the necessity for an implicit practical background of making some moves that are just okay. Things that would be put in a logical system, not in the forms of axioms, but in the form of rules.
\end{document}