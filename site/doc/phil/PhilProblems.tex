\documentclass[12pt,a4paper]{report}
\usepackage{tikz}
\usepackage{hyperref}
\usepackage{amsmath}
\usepackage{amssymb}
\usepackage{amsthm}
\usetikzlibrary{arrows,positioning}
\tikzset{ >=stealth', punkt/.style={ rectangle, rounded corners, draw=black, very thick, text width=6.5em,minimum height=2em, text centered}, pil/.style={ ->, thick, shorten <=2pt, shorten >=2pt,}}
\begin{document}
 \href{doc/phil/People.html}{Previous} 
 \href{doc/phil.html}{Up} 
 \href{doc/phil/PhilSituations.html}{Next} 
 \href{doc/phil/PhilProblems.pdf}{PDF} 
\title{Phil Problems}

\tableofcontents
This chapter is a collection of situations that are brought up as problems to be solved by various philosophers.

\part{doc/phil/PhilProblems/AgrippanTrilemma.html|Agrippan Trilemma}
There are only three ways of completing a proof:
\begin{itemize}
    \item The \emph{circular} argument, in which the proof of some proposition presupposes the truth of that very proposition
    \item The \emph{regressive} argument, in which each proof requires a further proof, ad infinitum
    \item The \emph{dogmatic} argument, which rests on accepted precepts which are merely asserted rather than defended
\end{itemize}

This is a dilemma for classical logic in particular; it could be addressed by introducing nonmonotonic logic (default and challenge). Some claims (e.g. first person observations) come with a default justification (which is not based on the justification of other claims). Yet it must be defended when challenged.
\part{doc/phil/PhilProblems/BradleysProblem.html|Bradleys Problem}
Suppose there is a relation. There is an additional relation between the relation itself and the relata. Vicious regress.

\part{doc/phil/PhilProblems/Universals.html|Universals}
It seems there is no problem thinking about what triangles and white things are, but what of triangularity and whiteness?

My thoughts: This seems to be an instance of the \href{doc/phil/PhilSituations/ToothPain}{tooth pain} issue. Noun-ness was for ordinary objects (with obvious ontological status) in an earlier game, but our creativity with language led us to nounify many other words, leading to `objects' with unclear ontological status.
\part{doc/phil/PhilProblems/WhattheTortoiseSaidtoAchilles.html|What the Tortoise Said to Achilles}

Lewis Caroll (1895)

Tortoise assumes $p$ and the conditional $p \implies q$, and Achilles tries to convince the Tortoise to accept $q$. He says that logic \emph{obliges} you to acknowledge $q$ in this case. Tortoise is willing to go along with this but demands that this rule be made explicit, so Achilles add an extra axiom $p \land (p \implies q) \implies q$. Achilles says that, now, you \emph{really} have to accept $q$ that we've written down $p$, $p \implies q$ and $p \land (p \implies q) \implies q$. But the Tortoise notes that, if that's really something logic obliges me to do, then it bears writing down ($p \land (p \implies q) \land (p \land (p \implies q) \implies q)$). Ad infinitum.

The most influential pragmatist work in the philosophy of logic, according to Brandom. The lesson: in any particular case, you can substitute a rule (that tells you you can go from this to that) with an axiom. But there have got to be some moves you can make without having to explicitly license them by a principle.  You've got to distinguish between 1.) premises from which to reason and 2.) principles in accordance with which to reason. This teaches an un-get-over-able lesson about the necessity for an implicit practical background of making some moves that are just okay. Things that would be put in a logical system, not in the forms of axioms, but in the form of rules.
\end{document}