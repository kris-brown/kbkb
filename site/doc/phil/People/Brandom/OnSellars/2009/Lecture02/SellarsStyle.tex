\documentclass[12pt,a4paper]{report}
\usepackage{tikz}
\usepackage{hyperref}
\usepackage{amsmath}
\usepackage{amssymb}
\usepackage{amsthm}
\usetikzlibrary{arrows,positioning}
\tikzset{ >=stealth', punkt/.style={ rectangle, rounded corners, draw=black, very thick, text width=6.5em,minimum height=2em, text centered}, pil/.style={ ->, thick, shorten <=2pt, shorten >=2pt,}}
\begin{document}

 \href{doc/phil/People/Brandom/OnSellars/2009/Lecture02.html}{Up} 
 \href{doc/phil/People/Brandom/OnSellars/2009/Lecture02/MaterialInference.html}{Next} 
 \href{doc/phil/People/Brandom/OnSellars/2009/Lecture02/SellarsStyle.pdf}{PDF} 
\title{Sellars Style}
% ORD 1
% GIST The 'mystery story' vs 'journalistic' styles of philosophical writing.

\begin{itemize}
\item `Mystery story' style:
    \begin{itemize}
        \item There's a problem, and many competing potential explanations
        \item These explanations engage each other dialectically
        \item Only at the end would you learn the philosopher's actual position
    \end{itemize}
\item `Journalistic' style:
    \begin{itemize}
        \item Tell them what you're going to tell them
        \item Tell them
        \item tell them what you told them
    \end{itemize}
\end{itemize}

Sellars philosophical style is more the former.
\end{document}