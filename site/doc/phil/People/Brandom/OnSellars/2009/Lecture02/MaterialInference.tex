\documentclass[12pt,a4paper]{report}
\usepackage{tikz}
\usepackage{hyperref}
\usepackage{amsmath}
\usepackage{amssymb}
\usepackage{amsthm}
\usetikzlibrary{arrows,positioning}
\tikzset{ >=stealth', punkt/.style={ rectangle, rounded corners, draw=black, very thick, text width=6.5em,minimum height=2em, text centered}, pil/.style={ ->, thick, shorten <=2pt, shorten >=2pt,}}
\begin{document}
 \href{doc/phil/People/Brandom/OnSellars/2009/Lecture02/SellarsStyle.html}{Previous} 
 \href{doc/phil/People/Brandom/OnSellars/2009/Lecture02.html}{Up} 
 \href{doc/phil/People/Brandom/OnSellars/2009/Lecture02/MaterialInferenceMainIdea.html}{Next} 
 \href{doc/phil/People/Brandom/OnSellars/2009/Lecture02/MaterialInference.pdf}{PDF} 
\title{Material Inference}
% ORD 2
% GIST An inference that is good NOT due to their logical form.
% TAG Def

Descriptive terms appear vacuously when in logically valid inferences and essentially when in \emph{material inferences}. You can turn a good material inference into a bad one by substituting some nonlogical vocabulary for different nonlogical vocabulary, but you cannot turn a logically valid inference into a bad one by the same means. Example: $P \land Q \implies P$ goes through regardless of what we substitute for $P$ and $Q$, but the material inference ``$a$ is red'' $\implies$ ``$a$ is colored'' will become false if we replace `colored' with `square'.

\end{document}