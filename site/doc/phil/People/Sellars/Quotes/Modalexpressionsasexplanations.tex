\documentclass[12pt,a4paper]{report}
\usepackage{tikz}
\usepackage{hyperref}
\usepackage{amsmath}
\usepackage{amssymb}
\usepackage{amsthm}
\usetikzlibrary{arrows,positioning}
\tikzset{ >=stealth', punkt/.style={ rectangle, rounded corners, draw=black, very thick, text width=6.5em,minimum height=2em, text centered}, pil/.style={ ->, thick, shorten <=2pt, shorten >=2pt,}}
\begin{document}
 \href{doc/phil/People/Sellars/Quotes/Manasrationalanimal.html}{Previous} 
 \href{doc/phil/People/Sellars/Quotes.html}{Up} 
 \href{doc/phil/People/Sellars/Quotes/Modalitiesandnorms.html}{Next} 
 \href{doc/phil/People/Sellars/Quotes/Modalexpressionsasexplanations.pdf}{PDF} 
\title{Modal expressions as explanations}
``To make firsthand use of modal expressions / subjunctive conditionals, is to
be about the business of explaining a state of affairs or justifying an
assertion." (cite CDCM).

Note we explain an (actual) state of affairs. We justify an assertion
(non-actual, in the space of reasons).

\end{document}