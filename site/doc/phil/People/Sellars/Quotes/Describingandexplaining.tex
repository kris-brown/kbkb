\documentclass[12pt,a4paper]{report}
\usepackage{tikz}
\usepackage{hyperref}
\usepackage{amsmath}
\usepackage{amssymb}
\usepackage{amsthm}
\usetikzlibrary{arrows,positioning}
\tikzset{ >=stealth', punkt/.style={ rectangle, rounded corners, draw=black, very thick, text width=6.5em,minimum height=2em, text centered}, pil/.style={ ->, thick, shorten <=2pt, shorten >=2pt,}}
\begin{document}
 \href{doc/phil/People/Sellars/Quotes/Community.html}{Previous} 
 \href{doc/phil/People/Sellars/Quotes.html}{Up} 
 \href{doc/phil/People/Sellars/Quotes/Describingtheworldwithoutmodality.html}{Next} 
 \href{doc/phil/People/Sellars/Quotes/Describingandexplaining.pdf}{PDF} 
\title{Describing and explaining}
``Although describing and explaining are distinguishable, they are also in an
important sense inseparable. The descripitive and explanatory resources of
language advance hand-in-hand."

\begin{itemize}
    \item  These two kinds of discursive activity, one can be describing in a
           particular act and not explaining (and vice-versa).
    \item Globally, they're only intelligible in terms of their relation to
          each other.
    \item The claim that matters: you couldn't have an  (autonomous discursive)
          language in use that had one and not the other.
    \item The reason is at least that in order to describe something you have
          to place it in a space of implications (i.e. the above quote)
\end{itemize}
\end{document}