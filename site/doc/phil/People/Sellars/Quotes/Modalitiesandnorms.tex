\documentclass[12pt,a4paper]{report}
\usepackage{tikz}
\usepackage{hyperref}
\usepackage{amsmath}
\usepackage{amssymb}
\usepackage{amsthm}
\usetikzlibrary{arrows,positioning}
\tikzset{ >=stealth', punkt/.style={ rectangle, rounded corners, draw=black, very thick, text width=6.5em,minimum height=2em, text centered}, pil/.style={ ->, thick, shorten <=2pt, shorten >=2pt,}}
\begin{document}
 \href{doc/phil/People/Sellars/Quotes/Modalexpressionsasexplanations.html}{Previous} 
 \href{doc/phil/People/Sellars/Quotes.html}{Up} 
 \href{doc/phil/People/Sellars/Quotes/Nondescriptiveconcepts.html}{Next} 
 \href{doc/phil/People/Sellars/Quotes/Modalitiesandnorms.pdf}{PDF} 
\title{Modalities and norms}
``The language of modalities is a transposed language of norms.''

%(Related to \ref{phil-quotes-section-describing-world-without-modality})

What is the connection between the (alethic) modal sense of \emph{must} and the
 normative sense of \emph{must}.

Carnap uses `transposed' to talk about an unquoted word used in a sentence as a
transposed mode of speech (about use of a quoted-word). E.g. ``Red is a
quality.'' vs `` `Red' is a one-place predicate.''

\end{document}