\documentclass[12pt,a4paper]{report}
\usepackage{tikz}
\usepackage{hyperref}
\usepackage{amsmath}
\usepackage{amssymb}
\usepackage{amsthm}
\usetikzlibrary{arrows,positioning}
\tikzset{ >=stealth', punkt/.style={ rectangle, rounded corners, draw=black, very thick, text width=6.5em,minimum height=2em, text centered}, pil/.style={ ->, thick, shorten <=2pt, shorten >=2pt,}}
\begin{document}
 \href{doc/phil/People/Sellars/Quotes/Exemplification.html}{Previous} 
 \href{doc/phil/People/Sellars/Quotes.html}{Up} 
 \href{doc/phil/People/Sellars/Quotes/Judgmentintheorderofexplanation.html}{Next} 
 \href{doc/phil/People/Sellars/Quotes/Issemanticspsychological.pdf}{PDF} 
\title{Is semantics psychological}
``The means of semantical statements is no more a psychological word than is
the ought of ethical statements or the must of modal statements."

I think of this in the following way: the $\square$ operator can be applied to
statements to make them `modal' (which could be alethic, deontic, or many other
things like ``Jones thinks that ..."). One such $\square$ is ``Semantically,
...'' (or, ``Literally, ...''). Ordinary statements made in practical life
 could be interpreted as having an implicit ``Pragmatically, ...'' (in fact,
 could we define \emph{pragmatics} to be that which play this role?)
\end{document}