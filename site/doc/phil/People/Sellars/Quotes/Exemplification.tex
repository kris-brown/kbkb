\documentclass[12pt,a4paper]{report}
\usepackage{tikz}
\usepackage{hyperref}
\usepackage{amsmath}
\usepackage{amssymb}
\usepackage{amsthm}
\usetikzlibrary{arrows,positioning}
\tikzset{ >=stealth', punkt/.style={ rectangle, rounded corners, draw=black, very thick, text width=6.5em,minimum height=2em, text centered}, pil/.style={ ->, thick, shorten <=2pt, shorten >=2pt,}}
\begin{document}
 \href{doc/phil/People/Sellars/Quotes/Describingtheworldwithoutmodality.html}{Previous} 
 \href{doc/phil/People/Sellars/Quotes.html}{Up} 
 \href{doc/phil/People/Sellars/Quotes/Issemanticspsychological.html}{Next} 
 \href{doc/phil/People/Sellars/Quotes/Exemplification.pdf}{PDF} 
\title{Exemplification}
``Exemplification is a quasi-semantical relation, and it (and universals
\footnote{redness, lionhood}) is in the world only in that broad sense in which the
world includes linguistic norms and roles viewed from the standpoint of a
fellow paritipant."

\begin{itemize}
    \item This plays off Carnap's notion of `quasi-syntactical'.
    \item There's the narrow view of `the world', that's the world that science
          is the measure of all things. And the broader view of the world.
\end{itemize}

\end{document}