\documentclass[12pt,a4paper]{report}
\usepackage{tikz}
\usepackage{hyperref}
\usepackage{amsmath}
\usepackage{amssymb}
\usepackage{amsthm}
\usetikzlibrary{arrows,positioning}
\tikzset{ >=stealth', punkt/.style={ rectangle, rounded corners, draw=black, very thick, text width=6.5em,minimum height=2em, text centered}, pil/.style={ ->, thick, shorten <=2pt, shorten >=2pt,}}
\begin{document}
 \href{doc/phil/People/Sellars/Quotes/Againstphenomenalism.html}{Previous} 
 \href{doc/phil/People/Sellars/Quotes.html}{Up} 
 \href{doc/phil/People/Sellars/Quotes/Describingandexplaining.html}{Next} 
 \href{doc/phil/People/Sellars/Quotes/Community.pdf}{PDF} 
\title{Community}
``To say that a person desired to do A, thought it his duty to do B, but was forced to do C, is not to describe him as one might describe a scientific specimin. One does indeed describe him, but one does something more. And it's this `something more' which is the irreducible core of the framework of persons. In what does this `something more' consist? To think of a featherless bird as a person is to think of it as a being with which  one is bound up in a network of duties. From this point of view, the irreducibility of the personal is the irreducibility of the \emph{ought} or the \emph{is}. But even more basic than this to think of a featherless biped as a person is to construe its behavior in terms of actual or potential membership in an embracing group, each member of which thinks of itself as a member of the group. Let's call such a group a \emph{community}."

Deep connection between the normative, the social, and the self-conscious.
\end{document}