\documentclass[12pt,a4paper]{report}
\usepackage{tikz}
\usepackage{hyperref}
\usepackage{amsmath}
\usepackage{amssymb}
\usepackage{amsthm}
\usetikzlibrary{arrows,positioning}
\tikzset{ >=stealth', punkt/.style={ rectangle, rounded corners, draw=black, very thick, text width=6.5em,minimum height=2em, text centered}, pil/.style={ ->, thick, shorten <=2pt, shorten >=2pt,}}
\begin{document}
 \href{doc/phil/People/Sellars/Quotes/Modalitiesandnorms.html}{Previous} 
 \href{doc/phil/People/Sellars/Quotes.html}{Up} 
 \href{doc/phil/People/Sellars/Quotes/Rulesarelived.html}{Next} 
 \href{doc/phil/People/Sellars/Quotes/Nondescriptiveconcepts.pdf}{PDF} 
\title{Nondescriptive concepts}
``Once the tautology `the world is described by descriptive concepts' is freed
from the idea that the business of all non-logical concepts is to describe, the
 way is clear to an ungrudging recognition that many expressions which
 empiricists have relegated to second-class citizenship in discourse are not
 inferior, they're just different."

\begin{itemize}
    \item \emph{Descriptive} concepts are not the only kinds of concepts. There
         is a temptation of empiricists in assimilating all expressions to
         descriptive expressions (such that anything not intelligible as
         descriptive is defective).
    \item The negation of this is called \emph{decriptivism}.
    \item The Tractatus was supremely descriptivist, yet made a key advance by
        noting that logical expressions play a different kind of role (prior to
        Tractatus, Russell would be looking into the world to try to see what
        distinguishes positive from negative facts, when he should not have
        been looking to the world).
    \item Sellars extends this even further. The later Wittgenstein saw
        language as playing an unsurveyable variety of roles.
    \item Most interesting philosophical concepts come with a ``-ing'' ``-ed''
        distinction that's crucial. Is a justification an act of justifying vs
        what's justified (related Agrippean trilemma). Is an experience an act
        of experiencing vs what's experienced.
\end{itemize}
\end{document}