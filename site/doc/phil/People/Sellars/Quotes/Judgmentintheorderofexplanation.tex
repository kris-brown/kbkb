\documentclass[12pt,a4paper]{report}
\usepackage{tikz}
\usepackage{hyperref}
\usepackage{amsmath}
\usepackage{amssymb}
\usepackage{amsthm}
\usetikzlibrary{arrows,positioning}
\tikzset{ >=stealth', punkt/.style={ rectangle, rounded corners, draw=black, very thick, text width=6.5em,minimum height=2em, text centered}, pil/.style={ ->, thick, shorten <=2pt, shorten >=2pt,}}
\begin{document}
 \href{doc/phil/People/Sellars/Quotes/Issemanticspsychological.html}{Previous} 
 \href{doc/phil/People/Sellars/Quotes.html}{Up} 
 \href{doc/phil/People/Sellars/Quotes/LabelingvsDescribing.html}{Next} 
 \href{doc/phil/People/Sellars/Quotes/Judgmentintheorderofexplanation.pdf}{PDF} 
\title{Judgment in the order of explanation}

``Kant was on the right track when he insisted that, just as concepts are
essentially (not accidentally) items which can occur in judgments, so judgments
(and therefore indirectly concepts) are items that occur essentially (not
accidentally) that occur in reasonings or arguments."

Turns a tradition (order of explanation: concepts -> judgments -> inferences)
on its head.

\end{document}