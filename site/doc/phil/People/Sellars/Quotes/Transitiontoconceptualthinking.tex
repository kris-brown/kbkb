\documentclass[12pt,a4paper]{report}
\usepackage{tikz}
\usepackage{hyperref}
\usepackage{amsmath}
\usepackage{amssymb}
\usepackage{amsthm}
\usetikzlibrary{arrows,positioning}
\tikzset{ >=stealth', punkt/.style={ rectangle, rounded corners, draw=black, very thick, text width=6.5em,minimum height=2em, text centered}, pil/.style={ ->, thick, shorten <=2pt, shorten >=2pt,}}
\begin{document}
 \href{doc/phil/People/Sellars/Quotes/Subjunctiveconditionalsasessential.html}{Previous} 
 \href{doc/phil/People/Sellars/Quotes.html}{Up} 

 \href{doc/phil/People/Sellars/Quotes/Transitiontoconceptualthinking.pdf}{PDF} 
\title{Transition to conceptual thinking}
``Anything which can properly be called `conceptual thinking' can occur only within a framework of conceptual thinking in terms of which it can be critized, supported, refuted (in short, evaluated). To be able to think is to be able to measure one's thoughts by standards of correctness, of relevance, of evidence (NB: in the space of justification). In this sense the diversified conceptual framework is a whole which (however sketchy) is prior to its parts, and cannot be construed as a coming to gether of parts which are conceptual in character. The conclusion is difficult to avoid that the transition from preconceptual patterns of behavior to conceptual thinking is a holistic one, a jump to a level of awareness that is irreducibly new. A jump which was the coming of the being of man." - Constrast with Wittegnsteining image of ``light dawning slowly on the whole". Sellars is worried about the mechanism of how this happens.

\end{document}