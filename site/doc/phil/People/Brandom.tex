\documentclass[12pt,a4paper]{report}
\usepackage{tikz}
\usepackage{hyperref}
\usepackage{amsmath}
\usepackage{amssymb}
\usepackage{amsthm}
\usetikzlibrary{arrows,positioning}
\tikzset{ >=stealth', punkt/.style={ rectangle, rounded corners, draw=black, very thick, text width=6.5em,minimum height=2em, text centered}, pil/.style={ ->, thick, shorten <=2pt, shorten >=2pt,}}
\begin{document}

 \href{doc/phil/People.html}{Up} 
 \href{doc/phil/People/Kant.html}{Next} 
 \href{doc/phil/People/Brandom.pdf}{PDF} 
\title{Brandom}

\tableofcontents

\part{doc/phil/People/Brandom/Antirepresentationalism.html|Antirepresentationalism}
A lecture series taught in 2020.
\chapter{doc/phil/People/Brandom/Antirepresentationalism/Lecture01.html|Lecture01}

\chapter{doc/phil/People/Brandom/Antirepresentationalism/Lecture02.html|Lecture02}

\chapter{doc/phil/People/Brandom/Antirepresentationalism/Lecture03.html|Lecture03}

\chapter{doc/phil/People/Brandom/Antirepresentationalism/Lecture04.html|Lecture04}

\chapter{doc/phil/People/Brandom/Antirepresentationalism/Lecture05.html|Lecture05}

\chapter{doc/phil/People/Brandom/Antirepresentationalism/Lecture06.html|Lecture06}

\chapter{doc/phil/People/Brandom/Antirepresentationalism/Lecture07.html|Lecture07}

\chapter{doc/phil/People/Brandom/Antirepresentationalism/Lecture08.html|Lecture08}

\chapter{doc/phil/People/Brandom/Antirepresentationalism/Lecture09.html|Lecture09}

\chapter{doc/phil/People/Brandom/Antirepresentationalism/Lecture10.html|Lecture10}

\chapter{doc/phil/People/Brandom/Antirepresentationalism/Lecture11.html|Lecture11}

\chapter{doc/phil/People/Brandom/Antirepresentationalism/Lecture12.html|Lecture12}

\chapter{doc/phil/People/Brandom/Antirepresentationalism/Lecture13.html|Lecture13}

\chapter{doc/phil/People/Brandom/Antirepresentationalism/Lecture14.html|Lecture14}

\part{doc/phil/People/Brandom/OnSellars.html|On Sellars}
Robert Brandom taught a recorded graduate seminar on Wilfred Sellars twice, in 2009 and and 2019. Notes on the recordings of these lectures are collected here.
\chapter{doc/phil/People/Brandom/OnSellars/2009.html|2009}

\section{doc/phil/People/Brandom/OnSellars/2009/Lecture01.html|Lecture01}
The first of many lectures by Robert Brandom on Wilfred Sellars, delivered on September 2, 2009.

\subsection{doc/phil/People/Brandom/OnSellars/2009/Lecture01/HistoricalKantContext.html|Historical Kant Context}
Kant was not in favor within analytic philosophy when Sellars began. ``One cannot open the door enough for Kant to get through while being able to slam it shut before Hegel gets through.'' (He's too interesting of a reader, and Hegel was the great bad of Anglophone philosophy). Though it's also ironic since Kant is incredibly analytical and science-driven.

Four ideas of Kant that mattered to Sellars:
\begin{enumerate}
\item \textbf{Kant's normative turn}:
\begin{itemize}
\item His normative understanding of discursive (i.e. relating to concepts) practice. Kant saw that what distinguishes judgments/intentional behavior from habitual behavior is that there are things that the agents are in a distintive sense \emph{responsible} for. this point is shared by the later Wittgenstein. The \href{doc/phil/Phil Situations/Childrens Game.html}{puzzles} that Wittgenstein offers us (along the way to trying to dissolve the presuppositions that make it puzzling) center around the normative significance of beliefs/desires/intentions.

\item The difference between us and it is not an \emph{ontological} distintion but rather a \emph{deontological} distinction. Downstream of this is many of Kant's innovations. The minimum unit of awareness/experience is the judgment (you need concepts already for that). The subjective form of judgment (the ``I think", the emptiest form of judgment) mark of "who is responsible for the judgment" vs the subjective form of judgment the mark of what you've made yourself responsible to. In virtue of having your made yourself responsible to what you say you're talking about, is what make it that you're representing (it sets the standards of correctness).
\end{itemize}

\item \textbf{Turning Rousseau's definition of freedom into demarcating the normative}: Rousseau said ``Obedience to a law that one has laid oneself is freedom''. Kant turned this around to distinguish constraint by norms from constraint by power. Where natural things are bound by rules, we are bound by our conceptions of rules (i.e. to the extent we acknowledge them). Our normative status depend on our attitudes.

\item \textbf{Pure concepts of the understanding}: In addition to concepts whose principle expressive job is to describe/explain empirical goings-on, there are concepts whose principle expressive job it is to make explicit the framework that makes description possible. These are known \textit{a priori}. Framework-explicating concepts. This is Kant's response to Hume, for how we can understand the modal force of laws in virtue of their non-modal description. The answer is in the description framework itself. The fact that there are necessarily relations that concepts have among another makes description possible (a concept being contentful at all requires it to have some necessary relations to other concepts). What Sellars means by `ushering philosphy from its Humean phase to its Kantian phase' is putting categories front and center. Trying to \emph{describe} the modal structure of the world or describe the space of possible worlds is to try to assimilate modal language into descriptivism, rather than seeing them as playing a different expressive role (Sellars saw Kant as putting this other option on the table).
\end{enumerate}. Difference between Humean thinking and Kantian thinking: do you take this categorial status in some form (rather than it being descriptive) - `laws of nature are not super-facts - you are not describing the world'. It's a transposed rule of inference.

Another Kantian idea: the distinction between phenomena and noumena. Kant radicalized the distinction between primary and secondary qualities (properties that are truly there vs properties that are due to us). He challenges us to divide the labor, what features is the world responsible for vs are we responsible for (e.g. the fact our theories are expressed in German/English)? This distinction lives in Sellars as the difference between the world (in the narrow sense) and the world (in the wider sense ... e.g. including norms that are only accessible from a participant's perspective).

\subsection{doc/phil/People/Brandom/OnSellars/2009/Lecture01/Sellarsquotes.html|Sellars quotes}
Some Sellars quotes on Describing, explaining, and justifying.

He doesn't begin with philosophically elaborated definition of describing, explaining, justifying. He takes these concepts as they come. He wants to do philosophy in a neutral / as close-to-practice way as possible.

\section{doc/phil/People/Brandom/OnSellars/2009/Lecture02.html|Lecture02}
This lecture was delivered on September 9, 2009.This lecture was delivered on September 9, 2009.

\subsection{doc/phil/People/Brandom/OnSellars/2009/Lecture02/SellarsStyle.html|Sellars Style}
% ORD 1
% GIST The 'mystery story' vs 'journalistic' styles of philosophical writing.

\begin{itemize}
\item `Mystery story' style:
    \begin{itemize}
        \item There's a problem, and many competing potential explanations
        \item These explanations engage each other dialectically
        \item Only at the end would you learn the philosopher's actual position
    \end{itemize}
\item `Journalistic' style:
    \begin{itemize}
        \item Tell them what you're going to tell them
        \item Tell them
        \item tell them what you told them
    \end{itemize}
\end{itemize}

Sellars philosophical style is more the former.
\subsection{doc/phil/People/Brandom/OnSellars/2009/Lecture02/MaterialInference.html|Material Inference}
% ORD 2
% GIST An inference that is good NOT due to their logical form.
% TAG Def

Descriptive terms appear vacuously when in logically valid inferences and essentially when in \emph{material inferences}. You can turn a good material inference into a bad one by substituting some nonlogical vocabulary for different nonlogical vocabulary, but you cannot turn a logically valid inference into a bad one by the same means. Example: $P \land Q \implies P$ goes through regardless of what we substitute for $P$ and $Q$, but the material inference ``$a$ is red'' $\implies$ ``$a$ is colored'' will become false if we replace `colored' with `square'.

\subsection{doc/phil/People/Brandom/OnSellars/2009/Lecture02/MaterialInferenceMainIdea.html|Material Inference Main Idea}
% ORD 3
% GIST Material inference a more primitive notion than logical validity.

Sellars has two good ideas associated with material inference:
\begin{enumerate}
\item There \emph{are} some inferences that are good, not in virtue of their logical form.
\item Turn the above thought on its head and say: we can understand the content of these descriptive terms in terms of the materially good inferences they appear in (as premises or conclusions).
\end{enumerate}

%Historical aside:
%This idea is connected to Bolzano (a contemporary of Frege's), who thought about how abstraction was connected to reasoning. We look at a good inference and the class of substitutions under which it remains good. This allows us to move from an equivalence relation to the equivalence class. Frege takes this and uses this as the basis for his logic and metaphysics. his notion of a function is understood in terms of substitutions / inter-substitutions.
%Quine picked up this substitution methodology in some of his technical writings (e.g. his essay on Carnap on logical truth). Quine talks about logical things as ones in which all non-logical vocabulray occurs vacuously (not in a way that's essential to the goodness of inference).

By this account, material proprieties of inference are more fundamental than / conceptually prior to logical validity. You have to start with the notion of a good inference in order to understand what a logically good inference is.

\subsection{doc/phil/People/Brandom/OnSellars/2009/Lecture02/PhilofLogicAside.html|Phil of Logic Aside}
% ORD 4
% GIST Basic questions in philosophy of logic.

Aside: It took a while in the 20th century to realize that logic was not about
logical truth but rather about validity of inference. In classical logic can
you treat these interchangably, but not all (rough logics vs smooth logics -
whether the consequence relation can be determined by the set of all theorems).
Dummett has written about this issue.

What if we picked some other vocabulary (other than logical) to hold fixed? E.g.
substituting non-theological vocabulary for non-theological vocabulary.
``If justice is loved by the gods then justice is pious''. If no matter what we
substitute for justice the inference is good, we might say the sentence is true
in virtue of its theological form.

Philosophy of logic (See Quine's and Putnam's books both titled The Philosophy
of Logic) has two classic questions:
\begin{enumerate}
    \item a \emph{demarcation} question: what makes something logical vocabulary?
     \begin{itemize}
        \item Quine disallows second order quantifiers and the epilson of set
              theory, whereas Putnam allows them.
     \end{itemize}
    \item a \emph{correctness} question: which logical consequence relation to use:
      \begin{itemize}
      \item Classical? Intuitionistic? etc.
      \end{itemize}
\end{enumerate}

Sellars challenges this tradition (logical empiricism) by pointing out there is
a concern conceptually prior in the order of explanation to philosophy of
logic: materially good inferences.

\subsection{doc/phil/People/Brandom/OnSellars/2009/Lecture02/InferenceandMeaning.html|Inference and Meaning}
% ORD 5

A main argument of \textit{Inference and Meaning} (cite) is that any language that makes essential use of non-logical, descriptive vocabulary must be understand as having that vocabulary standing in materially good (rather than just logically good) inferences.

``Concepts as Involving Laws and inconceivable without them'' is the title of an unintelligible essay by Sellars (but the title is the thesis and intelligible).

Sellars answers the first major question by claiming logical vocabulary (more specifically, alethic modal vocabulary, about what's necessary and what's possible) has the expressive job of making explicit the material proprieties of inference that articulate the content of non-logical concepts. Frege is more explicit about this (that you can use this to distinguish logical vocabulary) than Sellars.

Dan Dennett argues that we have to take animals as grasping modus ponens because they treat some inferences as good and others as bad (see \ref{phil-scenarios-section-dog-disjunction}). You could make explicit the practical capacity the animal has using a statement of disjunctive syllogism, but Sellars would ask what is the surplus value of invoking that explicit expression? (Over simply describing what is the dog can do).

There have to be some practical moves you're just allowed to make without them having to take the form of explicit premises (see \ref{phil-problems-section-tortoise}). Sellars touches upon this in Reflections on Language Games. He talks about free/auxillary positions that you're always allowed to occupy. We could have the auxillary position $\forall x, \psi(x)\vdash \phi(x)$ which would license us to move from a position $\psi(a)$ to a position $\phi(a)$, but we could also encode this with position for each possible move ($\psi(a)\vdash \phi(a)$, $\psi(b)\vdash\phi(b)$, ...). He ways that we could imagine replacing positions with moves, but it's not possible to imagine all moves being replaced with positions (`a game without moves is Hamlet without the Prince of Denmark').

Sellars is addressing tradition that wants some small set of explicit principles in accordance with which to reason. Any inference you think is good that isn't derivable from that small set of principles (e.g. modus ponens) is actually an infamy (has some suppressed premises). This is early analytic philosophy's embrace of the new logic. Sellar's contrary view (radical at the time) is that actually the reasoning could be completely in order, just with material proprieties of reasoning. You can still give/ask for reasons and mean that $p$, but what the logic does is give you meta-linguistic control to talk about what is a good inference and say that $p \vdash q$ is a good inference. Sellars doesn't extrapolate from this that logic is an optional superstructure in our lives - we need to be able to think and talk about the goodness of inferences.

Brandom: logic is the organ of semantic self-consciousness. The set of concepts that lets us bring our endorsement of some inferences as good/bad (this endorsement as something that reasons can be given or asked for) into the game of giving/asking for reasons.

Example: $A\vdash B$ where $A$ is ``she asked me to hand her the dish towel" and $B$ is ``I shall hand her the dish towel''. Traditional analytic philosophy will call this an infamy since it does not explicitly state how her request engages my motivational structure. Sellars would want to say that this invocation of the desire makes explicit the endorsement of $A \vdash B$ rather than referring to some item of the world.

Sellars complains about Carnap treating logical consequence as a syntactically definable relation between sentences. Just writing down the rules under a heading `rules' instead of `axioms' isn't making explicit the normative force they have (it leaves out the rulishness - that a rule is a rule for \emph{doing} something). This is a subtle point that doesn't matter for many purposes, but Sellars believes it's important if you want to understand what's going on with reasoning.

Potential counterargument against sellars: subjunctive conditionals are not making explicit proprieties of inference, but in fact are descriptions about possible worlds. To address this, we note there are separate issues. Firstly, there's the question about whether it's intelligible to have descriptive vocabulary in play in a context where there's no counterfactual reasoning. E.g. Hume believes he understands empirical facts perfectly well (the cat is on the mat) but not statements about what's possible and necessary. But Kant saw that this isn't intelligble - you need to make a distinction about what's possible with the cat and what's not (it's possible for the cat to not be on the mat, but not possible for it to be larger than the sun) or else there's nothing you could say about the conctent of the concept of `cat' that I've got (it would be just a label). The second issue is the codifiability of proprieties of material inference by logical vocabulary: whether a possible worlds analysis is incompatible with seeing subjunctive conditionals as making properties of inference explicit. Sellars would like to see a possible worlds analysis that matches up.

``There's an important difference between logical / modal / normative predicates on the one hand, and such predicates as `red' on the other.'' There's nothing to the formal except their role in reasoning, indeed, their role and make as meta linguistics sort of making explicit something about the ground level. For the latter, he wants to argue that these predicates too are meaningful insofar as their role in reasoning, but it's less obvious.

``Red is a quality''. This conveys the same information as the syntactical sentence ``\textit{Red} is a one place predicate.'' \ref{phil-quotes-section-modalities-and-norms}. What you're doing in asserting that premise from which to reason (couched in modal vocabulary) is endorsing a principle in accordance with which to reason (couched in normative vocabulary).

We cannot completely identify modal and normative statements. When I say "copper melts at 1084 degrees" one makes a claim that is true even if ther were no reasoners (so it can't be a claim directly \emph{about} inferences being good). What it \emph{conveys} is about inferences, not what it \emph{says}. Likewise, I say ``The sun is shining'' while I convey ``I believe the sun is shining.''

It might help to make progress toward understanding the say/convey distinction (which Sellars admits he's not clear about) by distinguishing two flavors of inference:
\begin{enumerate}
\item semantic inference: good in virtue of the contents of the premises and the conclusion
\item pragmatic inference: good in virtue of what you're doing in asserting the premises or the conclusion.
 \begin{itemize}
 \item e.g. John says `your book is terrible' and I infer that he's mad at me
 \item Geech embedding distinction between the two: we look at whether we'd endorse ``My book is terrible, then John is mad at me". Because we wouldn't, we know the inference is pragmatic.

 \end{itemize}
\end{enumerate}
\subsection{doc/phil/People/Brandom/OnSellars/2009/Lecture02/SomeReflectionsonLanguageGames.html|Some Reflections on Language Games}
% ORD 7

Regulism (conceptual norms as a matter of explicit rules) vs regularism (norms in terms of actual regularities). These are identified with empiricist and rationalist approaches. (Kris: I also see prescriptivism and descriptivism in linguistics)

One purpose: ``I shall have a chief my present purpose if I've made plausible the idea that an organism might come to play a language game, that is to move from position to position, the system of moves and positions, and to do it because of the system without having to obey rules, and hence without having to be playing a meta language game.'' (Section 18)

He doesn't explicitly mention Wittgenstein (\ref{jokes-subsection-worst-philosopher}). (Other times he uses astrices to censor his name). Thinking about language in terms of rules is Kantian. His notion of norms was juridical/jurisprudential. A rule that enjoins the doing of an action A is a sentence in some language, which requires more rules to interpret (regress - how do we deal with it?). Kant identified this regress (A132/B171) - ``judgment is a peculiar talent that can be practiced only and not taught''. Which is using distinction between things that can be shown (by examples) vs taught. Wittgenstein addresses this regress in the late 100's of PI.

Rejecting mere conformity: If we just consider conforming to a rule rather than obeying a rule, there's no regress, but we lose the normativity.

``[Mere conformity people] claim that it's raining therefore the streets will be wet (when it isn't an infamatic abridgement of a formally valid argument) is merely the manifestation of a tendency to expect to see the wet streets when one finds it's raining. In this latter case, it's a manifestation of a process which at best can only simulate inference, since it's a habitual transition, and as such not governed by a principle or rule by reference to which can be characterized as valid or invalid. That Hume dignified the activation of an association with the phrase `causal inference' is about a minor flaw, they continue, in an otherwise brilliant analysis. It should, however, be immediately pointed out that before one has a right to say that what Hume calls `causal inference' really is an inference at all, but merely a habitual transition from one thought to another. And contrast that with in this context, the genuine logical inferences which are, one must pay the price of showing just how logical inference is something more than a mere habitual transition empiricists in the human tradition have rarely paid this price, a fact which is proved most unfortunate for the following reason. An examination of the history of the subject shows that those who have held that causal inference only simulates inference proper have been led to do so as a result of the conviction that if it were a genuine inference, the laws of nature, things that govern this would be discovered to us by pure reason.  As they're thinking of what's a good inference having to be something that's transparent merely by introspection in the way that the laws of logic are.'' (him making point about distinction of real inferences and mere associations. )

No distinction between correct and incorrect can be made by purely pointing to regularity - as Wittgenstein pointed out, you'll always find some regularity (there's some elegant rule that generates the sequence, for any arbitrary sequence). This is also called `disjunctiv-itis' or `gerrymandering objections'. After a debate between Dretsky and Fodor: we're trying to see what makes the word porcupine mean porcupine. When `porcupine' is used in an observational way, it's typically in response to porcupines. So can we use that regularity to understand what `porcupine' means? No, because of counterfactuals. If it happened that the porcupines we saw were almost always male, would the word mean male porcupine? Or if we look at dispositions, if they're disposed to also call echidnas porcupines (that's the disjunction), why not say that `porcupine' means porcupine or echidna?

``what's denied is the playing a game logically involves obedience to the rules of the game. And hence the ability to use the language to play the language game in which the rules are formulated.'' (page 29) Need a sense of playing the game stronger than conforming but weaker than having the rules in mind.

Metaphysicus suggests why not a non-linguistic awareness of the rules? This is its own regress.

``We've tacitly accepted so far and the dialectic dichotomy between merely conforming to the rules and obey. But surely this is a false dichotomy. Is there something in between, for it required us to suppose that the only way in which a complex system of activity can be involved in the explanation of the occurrence of a particular act is by the agent explicitly envisaging the system and intending its realization. And that's as much as to say that unless the agent conceives of the system, the conformity of his behavior to the system must be accidental. '' So what's needed he's saying, is going to be something that says, look, there's an explanation of why he conforms to the rules. That invokes the rules, but it doesn't invoke them by him being aware of them. One example of this is teleosemantics \ref{phil-situations-section-bee-waggles}.

The essential thing for Kant was a distinction between what was between acting according to a rule and acting according to a conception of a rule, or a representation (Vorstellung) of a rule. So, ordinary natural objects act according to rules, the laws of nature, but we act according to representations of rules / to conceptions of rules.

The explanation as to why I use the word `purple' for purple things, the rule plays a crucial part even if it is not in my head. It is in the teachers' heads (they're already in the language and can conceive of rules). So the rule is causally antecedent to my behavior, so I can be following the rule (without regress).

Related quesiton addressed here: \ref{phil-situations-section-classical-behaviorism}

How is it that I can apply a concept according to norms, to invoke a pre-linguistic awareness of universals, that's going to be a given. And the key thing is, because that pre-linguistic awareness is conceived of as providing \emph{reasons} for me to do this. It's not just that I've been \emph{trained} to respond to some physiological thing by doing it (that would be okay. That could be part of the the real explanation, the pattern governed explanation). It's that that pre-linguistic awareness provides reasons. And the claim is reasons are always making a move in a game that's making the inferential move. And the question is: what determines the norms that govern that? Then we're off on the on the regress, again, so we've got to have some story that doesn't have that form. The form of the argument against the myth of the given. It's the idea that the awareness that givenness provides something that can serve as a reason, but is itself not dependent on our having learned a language, having a conceptual scheme, and so on.

To do: understand language entry transitions and language exit transitions.

There is debate (but it should be more of a bigger deal, in Brandom's opinion) about what are the minimal features needed for one to have a discursive language practice. Brandom views logical language as optional (though the expressive power would be incredibly stunted, you could still give and ask for reasons). MacDowell and Sellars think otherwise, that there can't be discourse without a meta-language.

Sellars needs the notion of language to be something that evolves over time (rather than an instantaneous collection of rules) because we want the decision to make a material move to occur with in a language (one is not doing redescription in another language).
\section{doc/phil/People/Brandom/OnSellars/2009/Lecture03.html|Lecture03}
This lecture was delivered on September 16, 2009.

\section{doc/phil/People/Brandom/OnSellars/2009/Lecture04.html|Lecture04}
This lecture was delivered on September 23, 2009.

\section{doc/phil/People/Brandom/OnSellars/2009/Lecture05.html|Lecture05}
This lecture was delivered on September 30, 2009.

\section{doc/phil/People/Brandom/OnSellars/2009/Lecture06.html|Lecture06}
This lecture was delivered on October 7, 2009.

\section{doc/phil/People/Brandom/OnSellars/2009/Lecture07.html|Lecture07}
This lecture was delivered on October 14, 2009.

\section{doc/phil/People/Brandom/OnSellars/2009/Lecture08.html|Lecture08}
This lecture was delivered on October 21, 2009.

\section{doc/phil/People/Brandom/OnSellars/2009/Lecture09.html|Lecture09}
This lecture was delivered on October 28, 2009.

\section{doc/phil/People/Brandom/OnSellars/2009/Lecture10.html|Lecture10}
This lecture was delivered on November 11, 2009.

\section{doc/phil/People/Brandom/OnSellars/2009/Lecture11.html|Lecture11}
This lecture was delivered on November 18, 2009.

\section{doc/phil/People/Brandom/OnSellars/2009/Lecture12.html|Lecture12}
This lecture was delivered on December 2, 2009.

\chapter{doc/phil/People/Brandom/OnSellars/2019.html|2019}

\section{doc/phil/People/Brandom/OnSellars/2019/AlethicModalityI.html|Alethic Modality I}
This lecture was delivered on October 16, 2019.

\section{doc/phil/People/Brandom/OnSellars/2019/AlethicModalityII.html|Alethic Modality II}
This lecture was delivered on October 23, 2019.

\section{doc/phil/People/Brandom/OnSellars/2019/BeingandBeingKnown.html|Being and Being Known}
This lecture was delivered in late November, 2019.

\section{doc/phil/People/Brandom/OnSellars/2019/Conclusion.html|Conclusion}
This lecture was delivered on December 4, 2019.

\section{doc/phil/People/Brandom/OnSellars/2019/EPMandPhenomenalism.html|EPM and Phenomenalism}
This lecture was delivered on October 9, 2019.

\section{doc/phil/People/Brandom/OnSellars/2019/EarlyWritingsI.html|Early Writings I}
This lecture was delivered on September 04, 2019.

\subsection{doc/phil/People/Brandom/OnSellars/2019/EarlyWritingsI/Lecture.html|Lecture}

Three historical currents:

England: Bertrand Russell and GE Moore. Introducing analytic philosophy against absolute idealism (championed by Bradley, who was influenced by Hegel-inspired T.H. Greene who was reacting negatively to dominant British empiricists). Sellars admired On Denoting.

American: At turn of the century, German idealism dominated (Josiah Royce more popular speaker than Williams James). Two movements recoiling against this: American pragmatism and "new realism/critical realism" (AKA trinonminalists, including Sellars' father, which lost to pragmatism and logical empiricism).

German/Austrian: Marbourg (natural science focus of Kant) -> Carnap + Vienna Circle. Three periods, Aufbau (radical reductive empiricism: all statements must be definable in terms of immediate experience + logical vocabulary). Carnap then lightens up (in response to CI Lewis) and says statements must at least be able to be supported by evidence that comes from experience (replacing biconditional EXPERIENCE<->THEORETICAL with just a conditional EXPERIENCE->THEORETICAL;' there's a surplus on the theoretical side).
In parallel, Frankfurt school (Adorono/Walter Benjamin/Habermas), concerned with culture (and with Marxist inflection).

Tools of the syntactical phase of logical empiricism not adequate to address all general philosophical problems - it was improved by the semantic dimensions. He wants to turn the crank again to add a pragmatic dimension.

What did Sellars see of value in the reductivist Carnap? Carnap quote: ``A symbol is introduced (or, if it is already in use, is subsequently legitimized) by determining under what conditions it is to be employed in the representation of a state of affairs. The introduction or legitimization of the word 'horse', for instance, comes about by determining the conditions which must hold if we're to call something a horse, hence through statement of the distinguishing features of a horse or the definition of horse. We say of the symbol (that has been introduced / legitimized in such a way that we think is at least capable of legitimization) that it designates a concept. So, the symbol of a concept is a rule-governed symbol. Whether it be defined or not. Its use should above all be rule-governed. The symbol should not be employed in any old arbitrary way, but rather, in a determinate consistent way. Uniformity in the mode of employemnet can be secured either by explicitly laying down rules or merely through constant habit, linguistic usage. We have not yet said anything about what a concept is, but only for what it is for a symbol to designated a concept and this is all that can be said with any precision. But it's also enough, for when talk of concepts is meaningful, it invariably addresses concepts designated by symbols or concepts that can in principle be so designated. And such talk is basically alaways about these symbols and the laws of their use. The formation of a concept consists in the establishment of the law concerning the use of the symbol it is a word in the representation in a state of affairs''. Link to sellars quote: ``Grasp of a concept is always mastery of the use of a word" Even in the Aufbau, Carnap thinks of rule-governedness of symbols being crucial to the meaning of concepts.

One way to understand the core program of analytical philosophy: the project of elaborating the meanings of a puzzling vocabulary in terms of a base vocabulary (unproblematic) with logical vocabulary. Naturalism (trying to describe intensions/norms in terms of natural science)/empiricism (trying to describe laws in terms of sense experience) as an example. (Kris: is ordinary language philosophy an example?) If it's not possible to elaborate, then the vocabulary is seens as defective in some way.

Brandom tries to synthesize this view with pragmatism as exemplified by Wittgenstein/Sellars.
\section{doc/phil/People/Brandom/OnSellars/2019/EarlyWritingsII.html|Early Writings II}
This lecture was delivered on September 11, 2019.

\section{doc/phil/People/Brandom/OnSellars/2019/EmpiricismandthePhilosophyofMind.html|Empiricism and the Philosophy of Mind}
This lecture was delivered on October 2, 2019.

\section{doc/phil/People/Brandom/OnSellars/2019/InferentialismandNormativity.html|Inferentialism and Normativity}
This lecture was delivered on September 25, 2019.

\section{doc/phil/People/Brandom/OnSellars/2019/Introduction.html|Introduction}
This lecture was delivered on August 28, 2019.

\section{doc/phil/People/Brandom/OnSellars/2019/NominalismI.html|Nominalism I}
This lecture was delivered on October 30, 2019.

\section{doc/phil/People/Brandom/OnSellars/2019/NominalismII.html|Nominalism II}
This lecture was delivered on November 5, 2019.

\section{doc/phil/People/Brandom/OnSellars/2019/PurePragmatics.html|Pure Pragmatics}
This lecture was delivered on September 16, 2019.

\section{doc/phil/People/Brandom/OnSellars/2019/ScientificRealism.html|Scientific Realism}
This lecture was delivered on November 13, 2019.

\end{document}