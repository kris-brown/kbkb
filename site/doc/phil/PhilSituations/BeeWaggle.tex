\documentclass[12pt,a4paper]{report}
\usepackage{tikz}
\usepackage{hyperref}
\usepackage{amsmath}
\usepackage{amssymb}
\usepackage{amsthm}
\usetikzlibrary{arrows,positioning}
\tikzset{ >=stealth', punkt/.style={ rectangle, rounded corners, draw=black, very thick, text width=6.5em,minimum height=2em, text centered}, pil/.style={ ->, thick, shorten <=2pt, shorten >=2pt,}}
\begin{document}

 \href{doc/phil/PhilSituations.html}{Up} 
 \href{doc/phil/PhilSituations/ChildrensGame.html}{Next} 
 \href{doc/phil/PhilSituations/BeeWaggle.pdf}{PDF} 
\title{Bee Waggle}
When a foraging bee discovers a supply of food, it returns to the hive and does
a waggle dance. The rest of the hive then flies out in a certain direction and
distance to locate the food. In some sense, the bee communicates the information
of the food supply to the hive.

Can't we explain why a particular bee on one
occasion does that by invoking the pattern that
it's an instance of? What would it mean to say of
a bee returning from a food source that it's
turnings and wiggling has occurred \emph{because}
they're part of a complete dance? This is related
to distinction of pattern governed vs rule
obeying behavior. (cite some reflections on
language games) Ruth milliken, Sellars student,
devotes her career to this, developing the field
of teleosemantics and writing about it in
Language, Thought, and Other Biological
Categories.

A simpler example than the bee one for this is
that beavers slap their tail when there's danger
and beavers flee when they hear another beaver
slap its tail.

The idea of teleosemantics is an evolutionary
sort of semantics. Acknowledge language use is
normative (make distinction between correct and
incorrect use). You have to draw the distinction
in the way that's compatible with any degree of
badness of the participants actually following
the rules. We look at a reproductive family, the
normal explanation (a tehcnial term) of tail
slapping is that, in the evolutionary history,
lives were saved by it. When the explanation of
the persistence of tail slapping turns on
particular eventsi n the past where things worked
well that way (expressed in terms of
counterfactuals - no tail slap, then species dies
out), then we can say its part of the proper
function (technical term) to perform that
behavior. This  solves some puzzle cases: it
allows us to say the proper function of sperm is
to fertilize eggs even if a vanishingly small
fraction of them actually do (because if they
hadn't fertilized eggs in the past, there
wouldn't be sperm now).

Well, this is the form of explanation for semantics, not a beaver tail slaps
that aren't involved in more complicated interactions than that. Because the
same thing can happen when the reproductive families are uses of words. Where
you can explain why, you know, we use the word Aristotle, as we do, if has
having a proper function of, in the end, referring to Aristotle, because if
people in the past had not used it, in particular ways, we wouldn't be using
it today. And similarly, for predicates, and so on. She says, These are not
biological norms. But we can understand words as having proper functions in the
same sense in which even in the merely biological case, we can understand things
as having proper functions. And we can understand them as having proper
signaling functions. So there's a proper function for producing these things
and a proper function for receiving them. There, the analogy of the of the tail
slaps is a good one.

But in order to get semantic cut to understand semantic content, we don't need
to use any principles of explanation that aren't already intelligible


\end{document}