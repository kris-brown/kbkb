\documentclass[12pt,a4paper]{report}
\usepackage{tikz}
\usepackage{hyperref}
\usepackage{amsmath}
\usepackage{amssymb}
\usepackage{amsthm}
\usetikzlibrary{arrows,positioning}
\tikzset{ >=stealth', punkt/.style={ rectangle, rounded corners, draw=black, very thick, text width=6.5em,minimum height=2em, text centered}, pil/.style={ ->, thick, shorten <=2pt, shorten >=2pt,}}
\begin{document}
 \href{doc/phil/PhilSituations/ClassicalBehaviorism.html}{Previous} 
 \href{doc/phil/PhilSituations.html}{Up} 
 \href{doc/phil/PhilSituations/ToothPain.html}{Next} 
 \href{doc/phil/PhilSituations/MontaigneDog.pdf}{PDF} 
\title{Montaigne Dog}
Montaigne is impressed that his dog, when chasing
a rabbit and coming to a fork, runs a little way
down one of the paths and smells no rabbit, then
immediately runs down the other fork of the path
without stopping to smell to check if the rabbit
went that way. (CITE?)

The dog is acting in accordance with the
disjunctive syllogism. Do we say that the dog
understands disjunction?
\end{document}