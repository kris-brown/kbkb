\documentclass[12pt,a4paper]{report}
\usepackage{tikz}
\usepackage{hyperref}
\usepackage{amsmath}
\usepackage{amssymb}
\usepackage{amsthm}
\usetikzlibrary{arrows,positioning}
\tikzset{ >=stealth', punkt/.style={ rectangle, rounded corners, draw=black, very thick, text width=6.5em,minimum height=2em, text centered}, pil/.style={ ->, thick, shorten <=2pt, shorten >=2pt,}}
\begin{document}
 \href{doc/phil/PhilSituations/MontaigneDog.html}{Previous} 
 \href{doc/phil/PhilSituations.html}{Up} 

 \href{doc/phil/PhilSituations/ToothPain.pdf}{PDF} 
\title{Tooth Pain}
Imagine a community that talked about having gold or silver in one’s teeth,
and extends that practice to talk about having pain in one’s teeth. If, as a
matter of contingent fact, the practitioners can learn to use the expression
‘in’ in the new way, building on (but adapting) the old, they will have
fundamentally changed the \emph{meaning} of ``in". This can be seen by the
fact that, in the old practice, it made sense to ask where the gold was before
it was in one’s tooth, whereas in the new practice asking where the pain was
before it was in the tooth can lead only to a distinctively philosophical kind
of puzzlement.

\cite{brandom2019some}

\bibliography{my}
\bibliographystyle{amsalpha}
\end{document}